%%%%%%%%%%%%%%%%%%%%%%%%%%%%%%%%%%%%%%%%%%%%%%%%%%%%%%%%%%%%%%%%%%%%%%%%%%%%%%%%
\section{Example Code}\label{sec:examplecode}

The example code for PHAML demonstrates calling the phaml\_xxxx modules from within the ice-sheet model and how to start new solutions.  This is the standard way most of the libraries for GLIMMER-CISM are used.  This requires loading everything the model requires and running through the full set of calculations on the ice-sheet.  The calls to PHAML will be one small part of the overall simulation process.  Therefore, the example code is demonstrative of the smaller piece of code that would be in a larger module.


\begin{framecode}{6in}
\begin{verbatim}

use phaml
use phaml_example
use glide_types
type(phaml_solution_type) :: phaml_solution
type(glide_global_type) :: cism_model

!initialize all variables needed
call phaml_init(cism_model,phaml_solution)

!does the evaluation and places the 
!solution in cism_model%phaml%uphaml
call phaml_evolve(cism_model,phaml_solution)

!close and free variables
call phaml_close(phaml_solution)

\end{verbatim}
\end{framecode}

This is an example of when only the solution is desired and no intermediate steps are needed.  If a nonlinear or relaxation type of simulation is required then it is possible to walk through the solution one step at a time like the example below.  All of the options in PHAML can be tweaked in the evolve procedures of each module if needed.

\begin{framecode}{6in}
\begin{verbatim}

use phaml
use phaml_example
use glide_types
type(phaml_solution_type) :: phaml_solution
type(glide_global_type) :: cism_model

!initialize all variables needed
call phaml_init(cism_model,phaml_solution)

!creates the mesh and sets initial conditions 
call phaml_setup(cism_model,phaml_solution)

!looping through timesteps
do while(time .le. model%numerics%tend)

    !copy old solution and do one iteration
    call phaml_nonlin_evolve(cism_model,phaml_solution)
    
    !get the solution and copy to desired variable
    call phaml_getsolution(phaml_solution, cism_model%phaml%uphaml)
end do

!close and free variables
phaml_close(phaml_solution)

\end{verbatim}
\end{framecode}
%%%%%%%%%%%%%%%%%%%%%%%%%%%%%%%%%%%%%%%%%%%%%%%%%%%%%%%%%%%%%%%%%%%%%%%%%%%%%%%%
\section{Standalone Code}

GLIMMER-CISM provides an excellent framework for doing small scale simulations by using a basic set of libraries.  This is a good way to work with PHAML as well since you can integrate it with GLIMMER-CISM for simple tests without the overhead of the entire model.  There is an simple example driver like this included in the libphaml source files aptly named \href{http://svn.berlios.de/svnroot/repos/glimmer-cism/glimmer-cism2/libphaml/simple\_phaml.F90}{simple\_phaml.F90}. 

%%%%%%%%%%%%%%%%%%%%%%%%%%%%%%%%%%%%%%%%%%%%%%%%%%%%%%%%%%%%%%%%%%%%%%%%%%%%%%%%
\section{Debugging Options}

PHAML provides options to hand running code over to a debugger so that slaves can be monitored separately from the master.  These are very useful when handling usermod variables or when using many slaves on different processors.  

When calling `phaml\_create' the parameter `spawn\_form=DEBUG\_SLAVE' can be passed and then slaves will spawn in a an xterm window with a debugger.  This requires compiling with the `-g' flag though, and will default to GDB for the debugger.  If a different debugger is desired you can set the `debug\_command' parameter to specify which to use.
