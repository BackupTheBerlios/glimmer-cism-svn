\section{Supplied mass-balance schemes}
\subsection{Overview}
The user is, of course, free to supply their own mass-balance model for use
with GLIDE. However, GLIMMER includes within it a annual positive-degree-day model
for mass balance, shortly to be augmented by a similar daily model and an
hourly energy balance model. This section gives details of how to configure
and call these models.
\subsection{Annual PDD scheme}
\label{ug.mbal.pdd_scheme}
The annual PDD scheme is contained in the f90 module \texttt{glimmer\_pdd},
and the model parameters are contained in the derived type
\texttt{glimmer\_pdd\_params}. Configuration data is contained in a standard
GLIMMER config file, which needs to be read from file before initialising the
mass-balance model. The model is initialised by calling the subroutine
\texttt{glimmer\_pdd\_init}, and the mass-balance may be calculated annually
by calling \texttt{glimmer\_pdd\_mbal}. 

\subsubsection{Example of use:}
\begin{verbatim}
  use glimmer_pdd
  use glimmer_config

  ...

  type(glimmer_pdd_params) :: pdd_scheme
  type(ConfigSection),pointer :: config

  ...

  call glimmer_pdd_init(pdd_scheme,config)

  ...

  call glimmer_pdd_mbal(pdd_scheme,artm,arng,prcp,ablt,acab)
\end{verbatim}
In the subroutine call to \texttt{glimmer\_pdd\_mbal}, apart from the
parameter variable \texttt{pdd\_scheme}, there are three input fields
(\texttt{artm}, \texttt{arng} and \texttt{prcp}), which are, respectively, the
annual mean air temperature, annual temperature half-range, and annual
accumulated precipitation fields. The final two arguments are output fields
--- annual ablation (\texttt{ablt}) and annual mass-balance
(\texttt{acab}). All arrays are of type \texttt{real(sp)}. Temperatures are
degrees Celcius, and precipitation, ablation and mass-balance are measured in
m of water equivalent.
%
%
%
\subsubsection{Day-degree calculation}
%
The greater part of the information held in the \texttt{glimmer\_pdd\_params}
derived type comprises a look-up table (the \emph{PDD table}). The model is
implemented this way for computational efficiency.

The table has two dimensions: mean annual air temperature ($T_a$)
(as the second index) and annual air temperature half range (i.e.,
from July's mean to the annual mean $\Delta T_a$) (as the first
index).  Following \emph{Huybrechts and others} [1991], daily air
temperatures ($T_a^\prime$) are assumed to follow a sinusoidal
cycle
\begin{equation}
    T_a^\prime = T_a + \Delta T_a \cos \left( \frac{2 \pi t}{A}
    \right) + \textbf{R}(0,\sigma)
\end{equation}
where $A$ is the period of a year and $R$ is a random fluctuation
drawn from a normal distribution with mean 0 $^\circ$C and
standard deviation $\sigma$ $^\circ$C. \emph{Huybrechts and
others} [1991] indicate that the number of positive degree days
(D, $^\circ$C days) for this temperature series can be evaluated
as
\begin{equation}\label{pdd}
    D = \frac{1}{\sigma \sqrt{2 \pi}}
    \int\limits_0^A
    \int\limits_0^{T_a^\prime+2.5\sigma}
    T_a \times \exp \left( \frac{-(T_a-T_a^\prime)^2}{2 \sigma^2} \right) dT
    dt
\end{equation}
where $t$ is time.  The table is completed by evaluating this integral using a
public-domain algorithm (Romberg integration), by {\it Bauer} [1961]. The
inner and outer integrals are coded as two subroutines
(\texttt{inner\_integral} and \texttt{pdd\_integrand}), which call the Romburg
integration recursively.

The main parameter needed is the assumed standard deviation of
daily air temperatures, which can be set in the configuration file (the
default is 5 $^\circ $C). 

The positive-degree days are then looked up in the table (as a function of
$T_a$ and $\Delta T_a$). We take care to check that this look up is in done
within the bounds of the table.  The final value of $P$ is determined using
bi-linear interpolation given the four nearest entries in the table to the
actual values of $T_a$ and $\Delta T_a$.

The remainder of the loop completes the calculation of the
ablation and accumulation given this value for $P$.
%
\subsubsection{Mass balance calculation} 
%
We use the following symbols: $a$ is total annual ablation; $a_s$
is potential snow ablation; $b_0$ is the capacity of the snowpack
to hold meltwater by refreezing; the total number of positive
degree days ($D$); day-degree factors for snow and ice ($f_s$ and
$f_i$); and the fraction of snowfall that can be held in the
snowpack as refrozen meltwater ($W_max$). Note that the
day-degree factors have been converted from ice to water
equivalents using the ratio of densities.

 First,
determine the depth of superimposed ice ($b_0$) that would have to
be formed before runoff (mass loss) occurs as a constant fraction
($W_{max}$) of precipitation ($P$)
\begin{equation}
    b_0=W_{max} P.
\end{equation}
Now determine the amount of snow melt by applying a constant
day-degree factor for snow to the number of positive day-degrees
\begin{equation}
    a_s=f_s D.
\end{equation}
We now compare the potential amount of snow ablation with the
ability of the snow layer to absorb the melt.  Three cases are
possible. First, all snow melt is held within the snowpack and no
runoff occurs ($a=0$).  Second, the ability of the snowpack to
hold meltwater is exceeded but the potential snow ablation is
still less than the total amount of precipitation so that
$a=a_s-b_0$. Finally, the potential snow melt is greater than the
precipitation (amount of snow available), so that ice melt ($a_i$)
has to be considered as well.  The total ablation is therefore the
sum of snow melt (total precipitation minus meltwater held in
refreezing) and ice melt (deduct from total number of degree days,
the number of degree days needed to melt all snowfall and convert
to ice melt)
\begin{equation}
    a=a_s + a_i = P - b_0 + f_i \left( D-\frac{P}{f_s} \right).
\end{equation}
We now have a total annual ablation, and can find total net mass
balance as the difference between the total annual precipitation
and the total annual ablation.

Note that this methodology is fairly standard and stems from a
series of Greenland papers by Huybrechts, Letreguilly and Reeh in
the early 1990s.
%
%
%
\subsubsection{Configuration}
The annual PDD scheme is configured using a single section in the
configuration file:
\begin{center}
  \tablefirsthead{%
    \hline
  }
  \tablehead{%
    \hline
    \multicolumn{2}{|p{0.98\textwidth}|}{\emph{\small continued from previous page}}\\
    \hline
  }
  \tabletail{%
    \hline
    \multicolumn{2}{|r|}{\emph{\small continued on next page}}\\
    \hline}
  \tablelasttail{\hline}
  \begin{supertabular}{|l|p{11cm}|}
%%%% 
    \hline
    \multicolumn{2}{|l|}{\texttt{[GLIMMER annual pdd]}}\\
    \hline
    \multicolumn{2}{|p{0.98\textwidth}|}{Specifies parameters for the PDD
    table and mass-balance calculation}\\
    \hline
    \texttt{dx} & Table spacing in the $x$-direction ($^{\circ}$C) (default=1.0)\\
    \texttt{dy} & Table spacing in the $y$-direction ($^{\circ}$C) (default=1.0)\\
    \texttt{ix} & Lower bound of $x$-axis ($^{\circ}$C) (default=0.0)\\
    \texttt{iy} & Lower bound of $y$-axis ($^{\circ}$C) (default=-50.0)\\
    \texttt{nx} & Number of values in x-direction (default=31)\\
    \texttt{ny} & Number of values in x-direction (default=71)\\
    \texttt{wmax} & Fraction of melted snow that refreezes (default=0.6) \\
    \texttt{pddfac\_ice} & PDD factor for ice (m day$^{-1}\,^{\circ}$C$^{-1}$)
    (default=0.008)\\
    \texttt{pddfac\_snow} & PDD factor for snow (m day$^{-1}\,^{\circ}$C$^{-1}$) (default=0.003)\\
  \end{supertabular}
\end{center}
\subsubsection{References}

\noindent Bauer (1961) \emph{Comm. ACM} \textbf{4}, 255.

\noindent Huybrechts, Letreguilly and Reeh (1991) \emph{Palaeogeography,
Palaeoclimatology, Palaeoecology (Global and Planetary Change)}
\textbf{89}, 399-412.

\noindent Letreguilly, Reeh and  Huybrechts (1991) \emph{Palaeogeography,
Palaeoclimatology, Palaeoecology (Global and Planetary Change)}
\textbf{90}, 385-394.

\noindent Letreguilly, Huybrechts  and  Reeh (1991) \emph{Journal of
Glaciology} \textbf{37}, 149-157.
%
%%%%%%%%%%%%%%%%%%%%%%%%%%%%%%%%%%%%%%%%%%%%%%%%%%%%%%%%%%%%%%%%%%%%%%%%%%%%%
% Daily PDD scheme
%%%%%%%%%%%%%%%%%%%%%%%%%%%%%%%%%%%%%%%%%%%%%%%%%%%%%%%%%%%%%%%%%%%%%%%%%%%%%
%
\subsection{Daily PDD scheme}
\label{ug.mbal.daily_pdd_scheme}
The other PDD scheme supplied with GLIMMER is a daily scheme. This is simpler than the annual scheme in that it does not incorporate any stochastic variations. The mass-balance is calculated on a daily basis, given the daily mean temperature and half-range, and assuming a sinusoidal diurnal cycle. Consequently, the firn model is more sophisticated than with the annual scheme, and includes a snow-densification parameterization.
%
\subsubsection{Configuration}
The daily PDD scheme is configured using a single section in the configuration file:
\begin{center}
  \tablefirsthead{%
    \hline
  }
  \tablehead{%
    \hline
    \multicolumn{2}{|p{0.98\textwidth}|}{\emph{\small continued from previous page}}\\
    \hline
  }
  \tabletail{%
    \hline
    \multicolumn{2}{|r|}{\emph{\small continued on next page}}\\
    \hline}
  \tablelasttail{\hline}
  \begin{supertabular}{|l|p{11cm}|}
%%%% 
    \hline
    \multicolumn{2}{|l|}{\texttt{[GLIMMER daily pdd]}}\\
    \hline
    \multicolumn{2}{|p{0.98\textwidth}|}{Specifies parameters for the PDD
    table and mass-balance calculation}\\
    \hline
    \texttt{wmax} & Fraction of melted snow that refreezes (default=0.6) \\
    \texttt{pddfac\_ice} & PDD factor for ice (m day$^{-1}\,^{\circ}$C$^{-1}$)
    (default=0.008)\\
    \texttt{pddfac\_snow} & PDD factor for snow (m day$^{-1}\,^{\circ}$C$^{-1}$) (default=0.003)\\
    \texttt{rain\_threshold} & Temperature above which precipitation is held to be rain ($^{\circ}$C) (default=1.0)\\
    \texttt{whichrain} & {\raggedright
      Which method to use to partition precipitation into rain and snow:\\
      \begin{tabular}{lp{7cm}}
        {\bf 1} & Use sinusoidal diurnal temperature variation \\
        2 & Use mean temperature only\\
      \end{tabular}}\\
    \texttt{tau0} & Snow densification timescale (s) (default=10 years)\\
    \texttt{constC} & Snow density profile factor C (m$^{-1}$) (default=0.0165)\\
    \texttt{firnbound} & Ice-firn boundary as fraction of density of ice (default=0.872)\\
    \texttt{snowdensity} & Density of fresh snow ($\mathrm{kg}\,\mathrm{m}^{-3}$) (default=300.0)\\
  \end{supertabular}
\end{center}
