%%%%%%%%%%%%%%%%%%%%%%%%%%%%%%%%%%%%%%%%%%%%%%%%%%%%%%%%%%%%%%%%%%%%%%%%%%%%%%%%
\subsection{Introduction}
This appendix presents some compilation notes for the Fortran source files, these can be found in the \href{http://developer.berlios.de/projects/glimmer-cism/}{Glimmer-CISM2} project on the Berlios SVN server.  These include the following files:
\href{http://svn.berlios.de/svnroot/repos/glimmer-cism/glimmer-cism2/libphaml/phaml\_user\_mod.F90}{phaml\_user\_mod.F90},
\href{http://svn.berlios.de/svnroot/repos/glimmer-cism/glimmer-cism2/libphaml/phaml\_support.F90}{phaml\_support.F90},
\href{http://svn.berlios.de/svnroot/repos/glimmer-cism/glimmer-cism2/libphaml/phaml\_pde.F90}{phaml\_pde.F90},
\href{http://svn.berlios.de/svnroot/repos/glimmer-cism/glimmer-cism2/libphaml/phaml\_example.F90}{phaml\_example.F90},
\href{http://svn.berlios.de/svnroot/repos/glimmer-cism/glimmer-cism2/libphaml/phaml\_example\_pde.F90}{phaml\_example\_pde.F90}, and 
\href{http://svn.berlios.de/svnroot/repos/glimmer-cism/glimmer-cism2/libphaml/simple\_phaml.F90}{simple\_phaml.F90}.

These instructions all assume a POSIX-compatible system.  Although all of the software can be compiled on other operating systems, this has not been attempted on any other OS and therefore is absent.  PHAML must be compiled after the graphics libraries if the OpenGL graphics are desired.  The graphic libraries are optional though and unnecessary for running the ice sheet model.  They are merely present for extra visual output if desired.


All projects require make to build.  For the following instructions, the assumption will be made that a global programs directory exists at /usr/bin.  If the target system does not have this, then it is necessary to register the installation directory with the system globally so that other programs can find it.
%%%%%%%%%%%%%%%%%%%%%%%%%%%%%%%%%%%%%%%%%%%%%%%%%%%%%%%%%%%%%%%%%%%%%%%%%%%%%%%%
\subsection{Triangle}
\subsubsection{Compiling/Installing}
Triangle must be compiled separate from PHAML and installed.  This process is straightforward.  A program called showme is also compiled and can be installed.  It allows you to view the mesh files that Triangle generates which is useful for testing purposes.

\begin{framecode}{6in}
\begin{verbatim}

make
cp triangle /usr/bin
cp showme /usr/bin

\end{verbatim}
\end{framecode}

%%%%%%%%%%%%%%%%%%%%%%%%%%%%%%%%%%%%%%%%%%%%%%%%%%%%%%%%%%%%%%%%%%%%%%%%%%%%%%%%
\subsection{PHAML}

\subsubsection{Getting PHAML}
PHAML can be downloaded as an archive from the website \cite{PHAML:website}, and then unarchived into a directory anywhere on the system.

\subsubsection{Compiling PHAML}

PHAML is relatively easy to compile if all the library dependencies are satisfied.  The list of dependencies as well as instructions for additional libraries are at the end of this section.  The PHAML user guide \cite{phamldoc} has an excellent section covering setting up some of these libraries and dependencies as well.  The Quickstart guide is a must read before attempting any serious PHAML work.  A lot more detail is also provided on individual software packages that can be used and the benefits they can provide.  These instructions will now assume that the minimum requirements are met.

First the document `mkmkfile.sh' needs to be edited in the root directory of the PHAML source folder.  The following items must be set correctly: DEFAULT\_PHAML\_ARCH, \\ DEFAULT\_PHAML\_OS, DEFAULT\_PHAML\_F90, DEFAULT\_PHAML\_C, \\ DEFAULT\_PHAML\_PARLIB, DEFAULT\_PHAML\_BLAS, and DEFAULT\_PHAML\_LAPACK.  
All other variables can be left to their default value.  The file `mkmkfile.sh' lists the available options for each of these as well.  A brief overview of these are in the dependencies section below.  Once these variables have all been correctly set the file can be closed and run as a script.  As the name suggests, this scripts generates the makefile needed to compile the project with make.

Now just invoke make and it should compile provided things were set correctly.

\begin{framecode}{6in}
\begin{verbatim}

./mkmkfile.sh
make

\end{verbatim}
\end{framecode}

If the code does not compile then either the configuration file is incorrect or a dependency is unsatisfied.


\subsubsection{PHAML Dependencies}
These are required dependencies in order to compile and use PHAML.  
\begin{itemize}
    \item \textbf{POSIX compliance} - An operating system must be Unix compatible in order to properly work with PHAML.  Ubuntu was used for all test work.
    \item \textbf{Make} - The compile system. 
    \item \textbf{A Fortran Compiler} - Most Fortran compilers will work.  GFortran was the compiler it was tested against.
    \item \textbf{A C Compiler} - CC or GCC.  GCC was used.
    \item \textbf{MPI Library} - An MPI server must be running in order for PHAML to communicate with its subprocesses.  Openmpi was chosen for all tests.
    \item \textbf{BLAS} - Compiled from source.
    \item \textbf{LAPACK} - Compiled from source.
    \item \textbf{TRIANGLE} - Compiled from source.  Instructions are above.
\end{itemize}
 

\subsubsection{PHAML Additional Libraries}
PHAML covers all additional libraries in the user guide.  The only extra libraries used in this project were the graphic libraries which have a separate section below with instructions.
\begin{itemize}
    \item \textbf{OpenGL} - The main graphics library.
    \item \textbf{GLUT} - The OpenGL Utility kit.
    \item \textbf{F90GL} - This is a custom interface library so that OpenGL can be called from Fortran.
\end{itemize}

\subsection{Installing PHAML}
PHAML can be installed anywhere on the system.  For a system wide install, using the `opt' folder is convenient, and will be assumed for the instructions.  The PHAML home directory is the top level directory that should have a lib and a modules directory inside of it.  Then the global system variables need to be set.

\begin{framecode}{6in}
\begin{verbatim}

export PHAML_OS=linux
export PHAML_HOME=/opt/phaml

\end{verbatim}
\end{framecode}

\subsection{Including Graphics with PHAML}
\subsubsection{GLUT}

GLUT can be downloaded from the official site or a version can be downloaded from PHAML's website \cite{PHAML:website} that is guaranteed to work with PHAML.  For building this system the one from PHAML was used and it is suggested that this is followed.

Compiling Glut can be somewhat problematic since it depends on OpenGL to compile.  Specifically, it was unable to compile on a Ubuntu system with an ATI graphics card because the open source ATI drivers did not include the older SGIX functions which were needed.  This is largely a vendor issue since the graphics card manufacturer usually supplies the drivers for the card.  When an Nvidia card was used GLUT compiled quickly with no issue since their proprietary drivers included all necessary functions.

The first thing to do is to open Imakefile and remove "test" and "prog" from the SUBDIRS variable.  It should only say ``SUBDIRS = lib".  Then you can run mkmkfiles.imake and compile the project.
\begin{framecode}{6in}
\begin{verbatim}

./mkmkfiles.imake
make

\end{verbatim}
\end{framecode}

Installing GLUT is essential in order to compile F90GL and PHAML with graphics.  This includes installing the libraries and the source header files.  Where OpenGL is installed on a system varies for each operating system.  If the development files for OpenGL are installed on the system, then all the GLUT header files in include/GL should be copied to that OpenGL source directory.  On the test system this was /usr/include/GL.  The all the compiled GLUT libraries need to be copied into the global dynamic library directory.  On most UNIX based systems this is /usr/lib.  Depending on how OpenGL libs are detected the libs might need to also be copied to an X11 directory.


\subsubsection{F90GL}
F90GL can be downloaded from the official site which is the same as PHAML's website \cite{PHAML:website}.  The package can then be unarchived and compiled anywhere on the system.

Compiling F90GL is not that complicated, but it takes more time to set up than GLUT does.  The package includes custom compile scripts based on architecture, operating system, Fortran compiler, and OpenGL version.  The file that lists what the abbreviations are for is `mf\_key'.  Simply use the file that relates to the intended system and then run make while specifying the system.  The other important note is that the script may need to be modified to point to the correct GLUT libraries that were just compiled.  

\begin{framecode}{6in}
\begin{verbatim}

make -f mflum2

\end{verbatim}
\end{framecode}

%%%%%%%%%%%%%%%%%%%%%%%%%%%%%%%%%%%%%%%%%%%%%%%%%%%%%%%%%%%%%%%%%%%%%%%%%%%%%%%%
\section{GLIMMER-CISM}

There is fairly good documentation for working with GLIMMER on different platforms and therefore the documentation presented here will be focused on compiling GLIMMER-CISM from the repositories as it was done on the test build.

\subsection{Compiling GLIMMER-CISM}
Compiling GLIMMER is fairly straightforward and relies on the MAKE build system like the other projects have.  The dependencies are listed below and should be satisfied before attempting to compile GLIMMER-CISM itself.  Once these are installed GLIMMER-CISM2 can be checked out from the repositories. \citeauthor{cism:website}

\subsubsection{GLIMMER-CISM Dependencies}
    \begin{itemize}
        \item \textbf{Autoconf} - Tool used in the build system.
        \item \textbf{Make} - The compile system. 
        \item \textbf{A Fortran Compiler} - Most Fortran compilers should work.  GFortran was the compiler it was tested against.
        \item \textbf{A C Compiler} - CC or GCC.  GCC was used.
        \item \textbf{NetCDF} - The libraries for reading a NetCDF file.
        \item \textbf{Python} - Many build scripts and drivers use python scripting.
    \end{itemize}

\subsubsection{GLIMMER-CISM Additional Libraries}
GLIMMER-CISM is a very robust system with the ability to add several different solvers, libraries, and components.  They will not be listed here, but more information can be found in the user guide. \cite{glimmerdoc}

In general compiling GLIMMER-CISM proceeds as such:
\begin{itemize}
    \item Check out the project \href{http://svn.berlios.de/svnroot/repos/glimmer-cism/glimmer-cism2/}{GLIMMER-CISM2} using svn or get an archived file of the project for download.
    \item Open a terminal in the directory of the project.
    \item Run the following commands:

\begin{framecode}{6in}
\begin{verbatim}

./bootstrap 
./configure --with-netcdf=/usr FCFLAGS="-DNO_RESCALE 
            -O3 -pedantic-errors -fbounds-check" 
make  

\end{verbatim}
\end{framecode}

\end{itemize}
The ``--with-netcdf=/usr" can be omitted if the build system can find NetCDF in it's default location on the system.
\citep{cism:website}

\subsection{Installing GLIMMER-CISM}
Provided everything compiled correctly the build system has a built-in method for installing the binaries on the system.  Root privileges may be required. 
\begin{framecode}{6in}
\begin{verbatim}

make install

\end{verbatim}
\end{framecode}
%%%%%%%%%%%%%%%%%%%%%%%%%%%%%%%%%%%%%%%%%%%%%%%%%%%%%%%%%%%%%%%%%%%%%%%%%%%%%%%%
\section{Compiling PHAML With GLIMMER-CISM}
There are a few changes to the build system in order to provide the ability for PHAML to compile with GLIMMER-CISM as well as a few additional options that are available to debug the system.

%--with-phaml=/phaml
%--with-phaml-graphics=/f90gl do I need glut specified
After bootstrapping GLIMMER-CISM the same commands need to be run but now specifying to use phaml, and if debugging the OpenGL graphics.  Note that the graphics are not needed and are merely for extra visualization during debugging, but there is a lot of overhead required in getting libraries working. 

In order to compile PHAML by itself with GLIMMER-CISM you want:
\begin{framecode}{6in}
\begin{verbatim}

./bootstrap 
./configure --with-netcdf=/usr --with-phaml=/opt/phaml 
            FCFLAGS="-DNO_RESCALE -O3 -pedantic-errors -fbounds-check"
make  
make install

\end{verbatim}
\end{framecode}

%"LDFLAGS= -lphaml"
In order to include the graphics as well you'll need to add the optional configure tags as well as the libraries to include.  The graphics tag needs the location of the F90GL files.  The OpenGL libraries should automatically be linked by a system-wide install.

\begin{framecode}{6in}
\begin{verbatim}

./bootstrap 
./configure --with-netcdf=/usr --with-phaml=/opt/phaml 
            --with-phaml-graphics=/opt/f90gl FCFLAGS="-DNO_RESCALE 
            -O3 -pedantic-errors -fbounds-check" 
make 
make install

\end{verbatim}
\end{framecode}

%"LDFLAGS=-lf90glut -lf90GLU -lf90GL -lglut -lGLU -lGL -lphaml" 
