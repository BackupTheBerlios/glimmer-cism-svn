\section{Namelist files}
%
This section contains information about the contents of the namelist files
used to configure GLIMMER.
%
\subsection{Top-level GLIMMER configuration file}
%
The top-level configuration file contains the following elements in the
following order:
\begin{center}
\begin{tabular}{l}
\texttt{timesteps} namelist \\
\texttt{file\_paras} namelist \\
List of instance-specific configuration files \\
\end{tabular}
\end{center}
%
\subsubsection {Possible \texttt{timesteps} namelist entry}
%
\begin{center}
\begin{tabular}{|c|c|c|l|}
\hline
Name & Type & Default & Description (units) \\
\hline
\hline
\texttt{tinc} & real & 1.0 & Main model time-step (years) \\
\hline
\end{tabular}
\end{center}
%
\subsubsection{Possible \texttt{file\_params} namelist entry}
\begin{center}
\begin{tabular}{|c|c|c|l|}
\hline
Name & Type & Default & Description (units) \\
\hline
\hline
\texttt{ninst} & integer & 1 & Number of ice model instances \\
\hline 
\end{tabular}
\end{center}
%
\subsubsection{List of instance-specific configuration files}
%
These filenames are given relative to the working directory of the climate
model. They must not be longer than 70 characters, though this may be altered
by changing the value of the parameter \texttt{fname\_length} in file
\texttt{glimmer\_global.f90}. The filenames should be given as plain text,
without quotation marks, and separated by newlines. 
%
\subsection{Instance-specific configuration files}
%
The instance-specific configuration files contain the following elements in
the following order:
\begin{center}
\begin{tabular}{l}
netCDF configuration file \\
\texttt{sizs} namelist \\
\texttt{prj} namelist \\
Name of sigma file \\
\texttt{opts} namelist \\
\texttt{nums} namelist \\
\texttt{pars} namelist \\
\texttt{dats} namelist \\
\texttt{cons} namelist \\
\texttt{forc} namelist \\
\end{tabular}
\end{center}
%
\subsubsection{netCDF configuration file}
%
As noted above, netCDF I/O is controlled by a separate configuration file. The configuration file is described on detail in Section \ref{ug.sec.ncconf}. Each model instance needs to be supplied with its own netCDF configuration file.
%
\subsubsection{Possible \texttt{sizs} namelist entries}
%
The \texttt{sizs} namelist specifies the model domain,
and may contain these elements:
%
\begin{center}
\begin{tabular}{|c|c|c|l|}
\hline
Name & Type & Default & Description (units) \\
\hline
\hline
\texttt{ewn} & integer & -- & Number of grid-points in the east-west
direction \\
\texttt{nsn} & integer & -- & Number of grid-points in the north-south
direction \\
\texttt{upn} & integer & 11 & Number of model levels \\
\hline
\end{tabular}
\end{center}
%
\subsubsection{Possible \texttt{prj} namelist entries}
%
The \texttt{prj} namelist specifies the parameters of the map projection, and
may contain these elements:
%
\begin{center}
\begin{tabular}{|c|c|c|l|}
\hline
Name & Type & Default & Description (units) \\
\hline
\hline
\texttt{p\_type} & integer & 1 & Type of projection: \\
 & & & 1) Lambert Equal Area \\
 & & & 2) Spherical polar \\
 & & & 3) Spherical stereographic (oblique) \\
 & & & 4) Spherical stereographic (equatorial) \\
\hline  
\texttt{lonc} & real & -- & Longitide of projection centre (degrees east) \\
\texttt{latc} & real & -- & Latitude of projection centre (degrees)\\
\texttt{cpx} & real & -- & Local $x$-coordinate of projection centre
(grid-points) \\
\texttt{cpy} & real & -- & Local $y$-coordinate of projection centre
(grid-points) \\
\hline
\end{tabular}
\end{center}
%
\subsubsection{Name of sigma file}
%
Conforms to the same rules as other filenames.
%
\subsubsection{Possible \texttt{opts} namelist entries}
%
The \texttt{opts} namelist specifies various ice model options, and may
contain these elements:
%
\begin{center}
    \tablefirsthead{%
    \hline
Name & Type & Default & Description (units)\\
    \hline
\hline}
  \tablehead{%
    \hline
    \multicolumn{4}{|p{0.98\textwidth}|}{\emph{\small continued from previous page}}\\
    \hline
Name & Type & Default & Description (units)\\
    \hline
\hline}
  \tabletail{%
    \hline
    \multicolumn{4}{|r|}{\emph{\small continued on next page}}\\
    \hline}
  \tablelasttail{\hline}


\begin{supertabular}{|c|c|c|p{8.5cm}|}
\texttt{whichtemp} & integer & 1 & {\raggedright
 Method of ice temperature calculation: \\
 \begin{tabular}{lp{7cm}}
  0 & Set column to surface air temperature \\
  1 & Do full temperature solution (also finds vertical velocity and apparent
  vertical velocity). \\
  2 & Set column to surface air temperature, but correct for melting-point
  effects. \\
 \end{tabular}}\\
\hline
\texttt{whichartm} & integer & 3 & {\raggedright
Method of calculation of surface air temperature:\\ 
 \begin{tabular}{lp{7cm}}
 0 & Linear function of surface elevation\\
 1 & Cubic function of distance from domain centre\\
 2 & Linear function of distance from domain centre\\
 3 & Greenland conditions (function of surface elevation and latitude) including forcing\\
 4 & Antarctic conditions (sea-level air temperature -- function of position)\\
 5 & Uniform temperature, zero range (temperature set in \texttt{cons} namelist) \\
 6 & Uniform temperature, corrected for height, zero range.\\
 7 & Use large-scale temperature and range.\\
 \end{tabular}}
\\
\hline
\texttt{whichthck} & integer & 4 & {\raggedright
Source of initial conditions: \\
\begin{tabular}{lp{7cm}}
1 & Read from file\\
2 & Set equal to one time-step of net accumulation (where positive)\\
3 & Stepped, linear function of distance from domain centre\\
4 & Read from file\\
5-7& Unknown\\
\end{tabular}}\\
\hline
\texttt{whichflwa} & integer & 0 & {\raggedright Method for calculating flow factor $A$:\\
\begin{tabular}{lp{7cm}}
0 & \emph{Patterson and Budd} relationship\\
1 & \emph{Patterson and Budd} relationship, with temperature set to $-10^{\circ}\mathrm{C}$ \\
2 & Set equal to $1\times 10^{-16}\,\mathrm{yr}^{-1}\,\mathrm{Pa}^{-n}$\\
\end{tabular}}\\
\hline
\texttt{whichisot} & integer & 1 & {\raggedright
Bedrock elevation: \\
\begin{tabular}{lp{7cm}}
0 & Fixed at input values\\
1 & Local function of ice loading history (ODE)\\
2 & Local function of ice loading history (ODE) with flexure\\
\end{tabular}}\\
\hline 
\texttt{whichslip} & integer & 4 & {\raggedright
Horizontal bed velocity: \\
\begin{tabular}{lp{7cm}}
0 & Linear function of gravitational driving stress\\
1-3 & Unknown\\
4 & Set to zero everywhere\\
\end{tabular}}\\
\hline
\texttt{whichbwat} & integer & 2 &{\raggedright
 Basal water depth: \\
\begin{tabular}{lp{7cm}}
0 & Calculated from local basal water balance\\
1 & as (0), including constant horizontal flow\\
2 & Set to zero everywhere\\
\end{tabular}}\\
\hline
\texttt{whichmarn} & integer & 0 &{\raggedright
 Ice thickness: \\
\begin{tabular}{lp{7cm}}
0 & Set thickness to zero if relaxed bedrock is more than certain water depth (??) \\
1 &  Set thickness to zero if floating \\
2 &  No action \\
\end{tabular}}\\
\hline
\texttt{whichbtrc} & integer & 1 & {\raggedright
Basal slip coefficient: \\
\begin{tabular}{lp{7cm}}
0 & \texttt{tanh} function of basal water depth\\
1 &  Set equal to zero everywhere\\
\end{tabular}}\\
\hline
\texttt{whichacab} & integer & 2 &{\raggedright
 Net accumulation: \\
\begin{tabular}{lp{7cm}}
0 & EISMINT moving margin \\
1 & PDD mass-balance model [recommended] \\
2 & Accumulation only\\
\end{tabular}}\\
\hline
\texttt{whichstrs} & integer & 2 & {\raggedright
Stress solution: \\
\begin{tabular}{lp{7cm}}
0 & Zeroth-order\\
1 & First-order\\
2 & Vertically-integrated first-order\\
3 & No action (use when velocity found elsewhere)\\
\end{tabular}}\\
\hline
\texttt{whichefvs} & integer & ? & ?? \\
\hline
\texttt{whichbabc} & integer & 0 &  {\raggedright Basal boundary condition? No options!}\\
\hline
\texttt{whichevol} & integer & 0 & {\raggedright
Thickness evolution method:\\
\begin{tabular}{lp{7cm}}
0 & Pseudo-diffusion approach \\
2 & Diffusion approach (also calculates velocities)\\
\end{tabular}}\\
\hline 
\texttt{whichwvel} & integer & 0 & Vertical velocities: \\
 & & & 0) Usual vertical integration \\
 & & & 1) Vertical integration constrained so that \\
 & & & upper kinematic B.C. obeyed \\
\hline 
\texttt{whichmask} & integer & 0 & ?? \\
\hline
\texttt{whichprecip} & integer & 0 & {\raggedright
Source of precipitation:\\
\begin{tabular}{lp{7cm}}
0 & Uniform precipitation rate (set internally at present)\\
1 & Use large-scale precipitation rate\\
2 & Use parameterization of \emph{Roe and Lindzen}\\
\end{tabular}}\\
\hline
\end{supertabular}
\end{center}
%
\subsubsection{Possible \texttt{nums} namelist entries}
%
\begin{center}
\begin{tabular}{|c|c|c|l|}
\hline
Name & Type & Default & Description (units)\\
\hline
\hline
\texttt{ntem}    & real & 1 & Length of temperature time-step as \\
 & & & a multiple of main timestep \\
\hline
\texttt{nvel}    & real & 1 & Length of velocity time-step as \\
 & & & a multiple of main time-step\\
\hline
\texttt{niso}    & real & 1 & Length of isostasy time-step as \\
 & & & a multiple of main time-step\\
\hline
\texttt{nout(3)} & real & (1,10,10) & Time between outputs for time-series, \\
 & & & 2D and 3D fields (years)\\
\hline
\texttt{nstr}    & real & & Start-time for 2D and 3D output (years) \\
\hline
\texttt{thklim} & real & 100 & Lower limit for 3D ice calculations (m?) \\
\hline
\texttt{mlimit} & real & -- & ? \\
\hline
\texttt{dew} & real & -- & Horizontal grid spacing in east-west direction (m)
\\
\hline
\texttt{dns} & real & -- & Horizontal grid spacing in north-south direction (m) \\
\hline
\end{tabular}
\end{center}
%
\subsubsection{Possible \texttt{pars} namelist entries}
%
\begin{center}
\begin{tabular}{|c|c|c|l|}
\hline
Name & Type & Default & Description (units)\\
\hline
\hline
\texttt{geot} & real & $-5\times 10^{-2}$ & Geothermal heat flux
($\mathrm{Wm}^{-2}$) \\
\hline
\texttt{fiddle} & real & 3.0 & Multiplier for flow factor \\
\hline
\texttt{airt(2)} & real & (-3.15,-0.01) & Air temperature parameterization
factors \\
 & & & (K, $\mathrm{K}\,\mathrm{km}^{-3}$) \\
\hline
\texttt{nmsb(3)} & real & (0.5, $1.05\times 10^{-5}$,  & Net accumulation
factors used in \\
 & & $4.5\times 10^{5}$) & combination with $\mathtt{whichthck}=2,3$ \\
 & & & ($\mathrm{m}\,\mathrm{yr}^{-1}$, $\mathrm{yr}^{-1}$, m) \\
\hline
\texttt{hydtim} & real & 1000 & 1) Basal hydrology time constant (When \\
 & & & $\mathtt{whichbwat}=0$) (yr)\\
 & & & 2) Basal hydrology advection velocity (When \\
 & & & $\mathtt{whichbwat}=1$) ($\mathrm{m}\,\mathrm{yr}^{-1}$)\\
\hline
\texttt{isotim} & real & 3000 & Isostasy time-constant (used in combination \\
 & & & with \texttt{whichisot} (yr) \\
\hline 
\texttt{bpar(5)} & real & (2.0, 10.0, 10.0, & Basal traction factors (used in \\
 & & 0.0, 1.0) & combination with \texttt{whichbtrc}). These describe the \\
 & & & form of the $B=\tanh(W)$ function: \\
 & & & (1) Width of tanh curve \\
 & & & (2) $W$ at midpoint of tanh curve (m) \\
 & & & (3) $B$ minimum ($\mathrm{m}\,\mathrm{yr}^{-1}\,\mathrm{Pa}^{-1}$) \\
 & & & (4) $B$ maximum ($\mathrm{m}\,\mathrm{yr}^{-1}\,\mathrm{Pa}^{-1}$) \\
 & & & (5) multiplier for marine sediments \\
\hline
\end{tabular}
\end{center}
%
\subsubsection{Possible \texttt{dats} namelist entries}
%
\begin{center}
\begin{tabular}{|c|c|c|l|}
\hline
Name & Type & Default & Description (units)\\
\hline
\hline
& & & \\
\hline
\end{tabular}
\end{center}
%
\subsubsection{Possible \texttt{cons} namelist entries}
%
\begin{center}
\begin{tabular}{|c|c|c|l|}
\hline
Name & Type & Default & Description (units)\\
\hline
\hline
\texttt{lapse\_rate} & real & -8.0 & Lapse rate used when adjusting \\
 & & & air temperature for elevation ($\mathrm{K}\,\mathrm{km}^{-1}$) \\
\hline
\texttt{precip\_rate} & real & 0.5 & Uniform precipitation rate,
\\
 & & & used in conjuction with \texttt{whichprecip}=0 \\
 & & & ($\mathrm{m}\,\mathrm{yr}^{-1}$ water equivalent) \\
\hline
\texttt{air\_temp} & real & -20.0 & Uniform surface air temperature, \\
 & & & used in conjunction with \texttt{whichsurftemp}=0,1 ($^{\circ}\mathrm{C}$) \\
\hline
\texttt{albedo} & real & 0.4 & Ice albedo \\
\hline
\end{tabular}
\end{center}
%
\subsection{Possible \texttt{forc} namelist entries}
%
\begin{center}
\begin{tabular}{|c|c|c|l|}
\hline
Name & Type & Default & Description (units)\\
\hline
\hline
\texttt{trun} & real & -- & Length of model run (years).\\
 & & & {\bf N.B.} This variable {\em does not} control the length of \\
 & & & the model run --- it is used in the case when \texttt{whichartm}=3 \\
 & & & to tell the forcing initialisation how long the run is going to be. \\
\hline
\end{tabular}
\end{center}