\section{Example Climate Drivers}
GLIMMER comes with three climate drivers of varying complexity:
\begin{enumerate}
 \item \texttt{simple\_glide}: an EISMINT type driver
 \item \texttt{eis\_glide}: Edinburh Ice Sheet driver.
 \item \texttt{libglint}: interface to GENIE
\end{enumerate}
These drivers are described in some detail here.

\subsection{EISMINT Driver}
\subsubsection{Configuration}
\begin{center}
  \tablefirsthead{%
    \hline
  }
  \tablehead{%
    \hline
    \multicolumn{2}{|p{0.98\textwidth}|}{\emph{\small continued from previous page}}\\
    \hline
  }
  \tabletail{%
    \hline
    \multicolumn{2}{|r|}{\emph{\small continued on next page}}\\
    \hline}
  \tablelasttail{\hline}
  \begin{supertabular}{|l|p{11cm}|}
%%%% 
    \hline
    \multicolumn{2}{|l|}{\texttt{[simple]}}\\
    \hline
    \multicolumn{2}{|p{0.98\textwidth}|}{EISMINT type configuration.}\\
    \hline
    \texttt{temperature} & (real(2)) \\
    \texttt{massbalance} & (real(3)) \\
    \hline
  \end{supertabular}
\end{center}

\newpage
\subsection{EIS Driver}
\subsubsection{Configuration}
\begin{center}
  \tablefirsthead{%
    \hline
  }
  \tablehead{%
    \hline
    \multicolumn{2}{|p{0.98\textwidth}|}{\emph{\small continued from previous page}}\\
    \hline
  }
  \tabletail{%
    \hline
    \multicolumn{2}{|r|}{\emph{\small continued on next page}}\\
    \hline}
  \tablelasttail{\hline}
  \begin{supertabular}{|l|p{11cm}|}
%%%% 
    \hline
    \multicolumn{2}{|l|}{\texttt{[EIS ELA]}}\\
    \hline
    \multicolumn{2}{|p{0.98\textwidth}|}{Mass balance parameterisation of the EIS driver. The Equilibrium Line Altitude is parameterised with $$z_{\text{ELA}} = a + b\lambda + c\lambda^2 + \Delta z_{\text{ELA}},$$ where $\lambda$ is the latitude in degrees north. The mass balance is then defined by $$M(z^\ast)=\left\{
    \begin{array}{ll}
      2M_{\text{max}}\left(\frac{z^\ast}{z_{\text{max}}}\right)-M_{\text{max}}\left(\frac{z^\ast}{z_{\text{max}}}\right)^2&\mbox{for}\quad z^\ast\le z_{\text{max}}\\
      M_{\text{max}}&\mbox{for}\quad z^\ast>z_{\text{max}}
    \end{array}
    \right.,
$$ where $z^\ast$ is the vertical distance above the ELA. }\\
    \hline
    \texttt{ela\_file} & name of file containing ELA variation with time, $\Delta z_{\text{ELA}}$\\
    \texttt{ela\_a} & ELA factor $a$\\
    \texttt{ela\_b} & ELA factor $b$\\
    \texttt{ela\_c} & ELA factor $c$\\
    \texttt{zmax} & The elevation at which the maximum mass balance is reached, $z_{\text{max}}$\\
    \texttt{bmax} & The maximum mass balance, $M_{\text{max}}$\\
    \hline
%%%% 
    \hline
    \multicolumn{2}{|l|}{\texttt{[EIS Temperature]}}\\
    \hline
    \multicolumn{2}{|p{0.98\textwidth}|}{Temperature is also assumed to depend on latitude and the atmospheric lapse rate. It is thus described by a polynomial:$$T(t)=\sum_{i=0}^N a_i(t)\lambda^i+bz.$$}\\
    \hline
    \texttt{temp\_file} & name of file containing temperature forcing time series.\\
    \texttt{order} & order of polynomial, $N$.\\
    \texttt{lapse\_rate} & lapse rate, $b$.\\
    \hline
%%%% 
    \hline
    \multicolumn{2}{|l|}{\texttt{[EIS SLC]}}\\
    \hline
    \multicolumn{2}{|p{0.98\textwidth}|}{Global sea--level forcing}\\
    \hline
    \texttt{slc\_file} & name of file containing sea--level change time series.\\
    \hline
  \end{supertabular}
\end{center}

\newpage
\subsection{GLINT Driver}
\subsubsection{Configuration}
\begin{center}
  \tablefirsthead{%
    \hline
  }
  \tablehead{%
    \hline
    \multicolumn{2}{|p{0.98\textwidth}|}{\emph{\small continued from previous page}}\\
    \hline
  }
  \tabletail{%
    \hline
    \multicolumn{2}{|r|}{\emph{\small continued on next page}}\\
    \hline}
  \tablelasttail{\hline}
  \begin{supertabular}{|l|p{11cm}|}
%%%% 
    \hline
    \multicolumn{2}{|l|}{\texttt{[GLINT projection]}}\\
    \hline
    \multicolumn{2}{|p{0.98\textwidth}|}{Projection info}\\
    \hline
    \texttt{projection} & (integer) type of projection: 1 --- Lambert Equal Area, 2 --- Spherical polar, 3 --- Spherical stereographic (oblique), 4 --- Spherical stereographic (equatorial)\\
    \texttt{lonc} & Longitide of projection centre (degrees east)\\
    \texttt{latc} & Latitude of projection centre (degrees)\\
    \texttt{cpx} & Local $x$-coordinate of projection centre (grid-points)\\
    \texttt{cpy} & Local $y$-coordinate of projection centre (grid-points)\\
    \texttt{std parallel} & \\
    \hline
%%%% 
    \hline
    \multicolumn{2}{|l|}{\texttt{[GLINT climate]}}\\
    \hline
    \multicolumn{2}{|p{0.98\textwidth}|}{GLINT climate parameterisation}\\
    \hline
    \texttt{temperature\_file} & \\
    \texttt{artm\_mode} & {\raggedright
      Method of calculation of surface air temperature:\\
      \begin{tabular}{lp{10cm}}
        0 & Linear function of surface elevation\\
        1 & Cubic function of distance from domain centre\\
        2 & Linear function of distance from domain centre\\
        3 & Greenland conditions (function of surface elevation and latitude) including forcing\\
        4 & Antarctic conditions (sea-level air temperature -- function of position)\\
        5 & Uniform temperature, zero range (temperature set in \texttt{cons} namelist) \\
        6 & Uniform temperature, corrected for height, zero range.\\
        7 & Use large-scale temperature and range.\\
      \end{tabular}}\\
    \texttt{precip\_mode} & {\raggedright
      Net accumulation: \\
      \begin{tabular}{lp{10cm}}
        0 & EISMINT moving margin \\
        1 & PDD mass-balance model [recommended] \\
        2 & Accumulation only\\
      \end{tabular}}\\
    \texttt{acab\_mode} & {\raggedright
      Source of precipitation:\\
      \begin{tabular}{lp{7cm}}
        0 & Uniform precipitation rate (set internally at present)\\
        1 & Use large-scale precipitation rate\\
        2 & Use parameterization of \emph{Roe and Lindzen}\\
      \end{tabular}}\\
    \hline
  \end{supertabular}
\end{center}