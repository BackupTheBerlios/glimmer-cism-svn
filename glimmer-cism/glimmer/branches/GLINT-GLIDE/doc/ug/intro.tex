\section{Introduction}

\subsection{What is GLIMMER?}

GLIMMER\footnote{GENIE Land-Ice Model with Multiply Enabled
Regions} is the land ice component of GENIE\footnote{Grid-ENabled
Integrated Earth-system model}. Its design is motivated by the desire to
create an ice modelling system which is easy to interface to a wide variety of
climate models, without 
the user having to have a detailed knowledge of its inner workings. This is
accomplished by providing a very well-defined interface, which allows access to
all the functionality required by the user. All model fields and
time-dependent parameters required by the ice model are passed to it through
the argument lists of the supplied subroutines. Initialisation data is
supplied through namelist files, while netCDF\footnote{\texttt{http://www.unidata.ucar.edu/packages/netcdf/index.html}} 
files are used for platform independent binary I/O. 
Currently, a suite of Matlab routines is provided for visualisation,
but these may be replaced in due course.

\subsection{What are `Multiply Enabled Regions'?}

The most distinctive feature of the GLIMMER framework is the ability to
run the ice model over several different regions of the globe, and to define
those regions at runtime. Each specified region is termed an \emph{instance}
of the ice model. Primarily, then, GLIMMER provides a uniform interface
between a global climate model and an arbitrary number of ice models. The
processes of down-scaling input variables, and subsequently aggregating and
up-scaling outputs is handled invisibly by GLIMMER, leaving the user solely
with the tasks of supplying input data and parameters, and handling outputs in
the manner appropriate to the problem being tackled. These techniques could be
applied to any surface model that you might want to run only over
particular regions of the globe.

\subsection{Can I use GLIMMER with my climate model?}

We hope so! The external interface of GLIMMER is designed to be quite
flexible, but certain assumptions have necessarily been made about the form
taken by input fields, etc. In general, these are derived from the
characteristics of the GENIE climate model, and the Reading IGCM, which is one
of atmospheric models available in GENIE. Nevertheless, the specified input
fields are chosen on the basis of their physical importance, rather than
because of their availability within a given atmospheric model. This may mean
that some pre-processing has to be done before fields may be passed to
GLIMMER, which clearly might have cost implications.

In order to use GLIMMER, the following is necessary:

\begin{itemize}
\item Global input fields must be supplied on a latitude-longitude
  grid. The grid does not have to be uniform in latitude, meaning that
  Gaussian grids may be used. Irregular grids (e.g. icosahedral grids) are not
  supported currently. 
\item The boundaries between global grid boxes should be located halfway
  between adjacent grid-points. This is not strictly true of a Gaussian grid,
  but we make that approximation at the moment. Functionality may be provided
  to deal with this in future versions, however.
\item In the global field arrays, latitude must be indexed from north to south
  -- i.e. the first row of the array is the northern-most one. Again, some
  flexibility might be introduced into this in the future.
\item The global grid must not have grid points at either of the poles.
\item The user must supply the bedrock topography for each of the ice model
  instances. These grids are read from netCDF files.
\item GLIMMER is written in FORTRAN 90/95, and may currently only be called from
  other FORTRAN code. Work will begin soon on componentization of
  all parts of GENIE, comprising the `wrapping' of components in Java,
  etc. This will make the source language irrelevant, apparently.
\end{itemize}
