\section{Isostatic Adjustment}
The ice sheet model includes simple approximations for calculating isostatic adjustment. These approximations depend on how the lithosphere and the mantle are treated. For each subsystem there are two models. The lithosphere can be described as a
\begin{description}
\item[\textbf{local lithosphere:}] the flexural rigidity of the lithosphere is ignored, i.e. this is equivalent to ice floating directly on the asthenosphere;
\item[\textbf{elastic lithosphere:}] the flexural rigidity is taken into account;
\end{description}
while the man is treated as a
\begin{description}
\item [\textbf{fluid mantle:}] the mantle behaves like a non-viscous fluid, isostatic equilibrium is reached instantaneously;
\item [\textbf{relaxing mantle}] the flow within the mantle is approximated by an exponentially decaying hydrostatic response function, i.e. the mantle is treated as a viscous half space.
\end{description}
See my PhD thesis for implementation details.