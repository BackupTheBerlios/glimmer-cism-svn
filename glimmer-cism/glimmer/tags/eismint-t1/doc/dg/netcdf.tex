\section{netCDF I/O}
The netCDF\footnote{\texttt{http://www.unidata.ucar.edu/packages/netcdf/}} library is used for platform independent, binary file I/O. GLIMMER makes use of the f90 netCDF interface. The majority of the source files are automatically generated from template files and a variable definition file using a python script.

Basic netCDF I/O works. The netCDF subsystem uses the configuration file API, described in Section \ref{dg.sec.config_file}. The two major missing aspects are CF\footnote{\texttt{http://www.cgd.ucar.edu/cms/eaton/cf-metadata/index.html}} compliance and storing parameters defining the geographic projection used.

\subsection{Data Structures}
Dimension and variable IDs are stored in the derived type \texttt{glimmer\_nc\_stat}. The ID of a particular netCDF variable can be found by using the corresponding index (variable name prefixed with \texttt{NC\_B\_}, e.g. the index of the ice thickness variable, \texttt{thk}, is \texttt{NC\_B\_THK} and the variable ID is then \texttt{varids(NC\_B\_THK)}. Meta data (such as title, institution and comments) is stored in the derived type \texttt{glimmer\_nc\_meta}.

Input and output files are managed by two separate linked lists. Elements of the input file list contain the number of available time slices and information describing which time slice(s) should be read. Output file elements describe how often data should be written and the current time.

\subsection{The Code Generator}
Much of the code needed to do netCDF I/O is very repetative and can therefore be automatically generated. The code generator, \texttt{generate\_ncvars.py}, is written in python and produces source files from a templeate \texttt{.in} and the variable definition file, see Section \ref{dg.sec.vdf}. The templates are valid source files, all the generator does is replace special comments with the code generated from the variable file. Each supported template file is handled by a separate class derived from a base class. For further information check the documentation of \texttt{generate\_ncvars.py}\footnote{run \texttt{pydoc generate\_ncvars.py}}.

\subsection{Variable Definition File}\label{dg.sec.vdf}
All netCDF variables are defined in a control file, \texttt{ncdf\_vars.def}. Variables can be modified/added by editing this file. The file is read using the python \texttt{ConfigParser} module. The format of the file is similar to Windows \texttt{.ini} files, lines beginning with \texttt{\#} or \texttt{;} or empty lines are ignored. A new variable definition block starts with the variable name in square brackets []. Variables are further specified by parameter name/value pairs which are separated by \texttt{:} or \texttt{=}. Parameter names and their meanings are summarised in Table \ref{dg.tab.vdf}. All parameter names not recognised by the code generator (i.e. not in Table \ref{dg.tab.vdf}) are added as variable attributes.

\begin{table}[htbp]
 \begin{center}
  \begin{tabular}{|l|p{10cm}|}
    \hline
    name & description \\
    \hline
    \hline
    \texttt{dimensions} & List of comma separated dimension names of the variable. C notation is used here, i.e. the slowest varying dimension is listed first.\\
    \hline
    \texttt{data} & The variable to be stored/loaded. The f90 variable is assumed to have one dimension smaller than the netCDF variable, i.e. f90 variables are always snapshots of the present state of the model. Variables which do not depend on time are not handled automatically. Typically, these variables are filled when the netCDF file is created.\\
    \hline
    \texttt{factor} & Variables are multiplied with this factor on output and divided by this factor on input. Default: 1.\\
    \hline
    \texttt{load} & Set to 1 if the variable can be loaded from file. Default: 0.\\
    \hline
    \texttt{units} & UDUNITS compatible unit string describing the variable units.\\
    \hline
    \texttt{long\_name} & A more descriptive name of the variable.\\
    \hline
    \texttt{standard\_name} & The corresponding standard name defined by the CF standard.\\
    \hline
  \end{tabular}
  \caption{List of accepted variable definition parameters.}
  \label{dg.tab.vdf}
 \end{center}
\end{table}
