\section{GLIMMER in practice -- an example}

\subsection{Initialising and calling}

The easiest way to learn how GLIMMER is used is by way of an
example. We assume that the GLIMMER code has been installed alongside the
climate model code, and can be compiled and linked successfully. Details of
how to achieve this may be found in the \texttt{COMPILE} file in the top-level
GLIMMER directory. 

Typically, GLIMMER will be called from the main program body of a
climate model. To make this possible, the compiler needs to be told to use the
GLIMMER code. Use statements appear at the very beginning of f90 program
units, before even \texttt{implicit none}:
%
\begin{verbatim}
  use glimmer_main
\end{verbatim}
%
The next task is to declare a variable of type \texttt{glimmer\_params}, which
holds everything relating to the model, including any number of ice-sheet
instances:
%
\begin{verbatim}
  type(glimmer_params) :: ice_sheet
\end{verbatim}
%
Before the ice-sheet model may be called from the climate model, it must be
initialised. This is done with the following subroutine call:
%
\begin{verbatim}
  call initialise_glimmer(ice_sheet,lats,lons,paramfile)
\end{verbatim}
%
In this call, the arguments are as follows:
%
\begin{itemize}
\item \texttt{ice\_sheet} is the variable of type \texttt{glimmer\_params}
 defined above;
\item \texttt{lats} and \texttt{lons} are one-dimensional arrays giving the
  locations of the global grid-points in latitude and longitude, respectively; 
\item \texttt{paramfile} is the file name of the top-level GLIMMER
  configuration file.
\end{itemize}
%
The contents of the configuration files will be dealt with later. Having
initialised the model, it may now be called as part of the main climate
model time-step loop:
%
\begin{verbatim}
    call glimmer(ice_sheet,time,temp,precip,zonwind,merwind,orog)
\end{verbatim} 
%
The arguments given in this example are the compulsory ones only; a number of
optional arguments may be specified -- these are detailed in the reference
section below. The compulsory arguments are:
%
\begin{itemize}
\item \texttt{ice\_sheet} is the variable of type \texttt{glimmer\_params}
 defined above;
\item \texttt{time} is the current model time, in hours;
\item \texttt{temp} is the daily mean $2\,\mathrm{m}$ global air temperature field, in
  $^{\circ}\mathrm{C}$;
\item \texttt{precip} is the global daily precipitation fields,
  in $\mathrm{mm}\,\mathrm{s}^{-1}$ (water equivalent, making no distinction
  between rain, snow, etc.);
\item \texttt{zonwind} and \texttt{merwind} are the daily mean global zonal and
  meridional components of the $10\,\mathrm{m}$ wind field, in
  $\mathrm{ms}^{-1}$;
\item \texttt{orog} is the global orography field, in $\mathrm{m}$.
\end{itemize}
%
For the positive degree-day mass-balance routine, which is currently the only
mass-balance model included with GLIMMER, the daily quantities given above are
necessary, and, as such, GLIMMER should be called once per day. With the
energy and mass-balance model currently being developed, hourly calls will be
necessary. 
%
\subsection{Finishing off}
%
After the desired number of time-steps have been run, GLIMMER may have some
tidying up to do. To accomplish this, the subroutine \texttt{end\_glimmer}
must be called:
%
\begin{verbatim}
  call end_glimmer(ice_sheet)
\end{verbatim}
%

\subsection{Parameter Files}
\subsubsection{Format}
%
All parameter configuration files in GLIMMER use a custom, easily
understood file format, similar to Windows \texttt{.ini} files. The format is
described in full in section \ref{dg.sec.config_file}. 

Global parameters, applicable to all instances of the ice model are contained
in the file specified by \texttt{paramfile} in the call to
\texttt{initialise\_glimmer}. An example configuration file is given below (it is
the file \texttt{example.glp}):
%
\begin{verbatim}
[parameters]
time-step = 1.0
instance filenames:  g_land.gli
\end{verbatim}
%
The file contains only one section, \texttt{parameters}, which has two
name-value pairs: \texttt{time-step} and \texttt{instance filenames}. The
first of these is the main ice model time-step, in years. The second is a list
of configuration files for each ice model instance. GLIMMER will 
create ice model instances for each of the filenames listed here, without the
need to explicitly specify how many instances there are to be.

For each instance of the model, a configuration file with instance-specific
parameters must be supplied. The names of these files are given in the main
GLIMMER configuration file, as described above. The configuration file for an individual
instance is fairly long, so only a summary of the different sections
contained in it is given below. For an example this kind of file, see
\texttt{g\_land.gli} in the \texttt{data} directory.

Sections in an instance-specific file:
%
\begin{itemize}
\item \texttt{[output]} Configuration of model output -- contains the name of
  the configuration file controlling netCDF I/O. In the case of the example file
  \texttt{g\_land.gli}, the netCDF control file is \texttt{g\_land.glw}; see
  Section \ref{ug.sec.ncconf}. In addition, the file specified here also
  contains the name of the input file used (this will be changed in future).
\item \texttt{[domain size]} Model grid parameters -- number of grid points in $x$ and
  $y$ dimensions, and number of levels in the vertical.
\item \texttt{[projection]} Details of the projection used in the instance -- type of
  projection, and necessary parameters.
\item \texttt{[sigma coordinates]} -- contains the name of a
  sigma-coordinates file (see below)
\item \texttt{[options]} Flags specifying various model integration options, such
  as which of various schemes to use, etc.
\item \texttt{[time-steps]} The lengths of the various model time-steps
  relative to the main time-step.
\item \texttt{[grid-lengths]} The grid-lengths of the ice model domain.
\item \texttt{[parameters]} Physical parameters for the model, such as isostatic
  relaxation timescale.
\item \texttt{[forcing]} Parameters used by the forcing which may be used to run
  the model in stand-alone mode.
\item \texttt{[constants]} Constants used in the interface with the global
  model.
\item \texttt{[PDD scheme]} Parameters for the positive degree-day mass
  balance scheme.
\end{itemize}
%
\subsubsection{The Sigma coordinate file}
%
The name of the file containg the sigma variable is specified in
the instance-specific configuration file. This file consists of a
list of numbers in ascending order, between 0.0 and 1.0, specifying the
heights of the model levels in sigma space. The number of entries must be the
same as the number of model levels specified. An example file is
\texttt{g\_land.gls}.

\subsubsection{Grid I/O control files}
The model requires at least one field as input -- the height of the bedrock
topography (including bathymetry, if appropriate). If the bedrock is not in a relaxed state, then
the height of the relaxed topography must also be supplied. The bedrock is assumed to be relaxed 
if no relaxed topography field is supplied. Furthermore, the uppser surface of the ice sheet at the 
start of the run can be specified. These must be supplied as fields in a
netCDF file, whose name is given in the configuration file specified in the
\texttt{[output]} section of the main instance configuration file (see
above). A sample input file is supplied (\texttt{g\_land.input.nc}) - use
\texttt{ncdump} to determine the names of variables, dimensions and attributes.

Model output is controlled by the file listed in the \texttt{[output]} section
of the main instance-specific configuration file. Each model instance can save
specific variables to different netCDF files with different time intervals.
%
\subsection{Restarts}
%
{\bf N.B. GLIMMER does not currently have a functioning restart
  mechanism. This documentation is provided for information only at this stage.}

GLIMMER will provide two routines to handle restarts,
\texttt{glimmer\_write\_restart}, and \texttt{glimmer\_read\_restart}. The
former writes the entire model state to a single file, while the latter will
restore the model state from a previously created
file. For example, \texttt{glimmer\_write\_restart} may be called as follows:
%
\begin{verbatim}
  call glimmer_write_restart(ice_sheet,25,'ice_sheet.restart')
\end{verbatim}
%
Here, \texttt{ice\_sheet} is the GLIMMER parameter variable refered to
previously, \texttt{25} is the logical file-unit to use, and
\texttt{'ice\_sheet.restart'} is the filename of the restart file. This
subroutine call may be made at any point, regardless of whether it is intended
to halt the integration imminently, or not. In order to recover the model
state, the following call to \texttt{glimmer\_read\_restart} would be made:
%
\begin{verbatim}
  call glimmer_read_restart(ice_sheet,25,'ice_sheet.restart')
\end{verbatim}
%
The arguments are the same as for the previous call. When restarting from a
file like this, it is not necessary to make a call to
\texttt{initialise\_glimmer}. Note also that there is no alternative restart mechanism
provided within the normal \texttt{glimmer} subroutine call -- all restarts
must be called explicitly.

Note also that \texttt{glimmer\_read\_restart} may not be called if
\texttt{initialise\_glimmer} has been called already. This is because there is
currently no mechanism for \texttt{glimmer\_read\_restart} to know whether the
relevant model arrays have already been allocated. If they have, and
\texttt{glimmer\_read\_restart} tries to reallocate them, a fatal
run-time error will probably be generated. It is hoped to address this problem
in a future release.
%