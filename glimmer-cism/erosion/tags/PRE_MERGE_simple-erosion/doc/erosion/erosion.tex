\section{Bedrock Erosion}
Sediments are produced at the ice base by eroding the hard bedrock. Here, sediment production depends on basal velocities (and thus basal thermal regime) and ice thickness:
\begin{equation}
  \dot{E} = -f|\vec{u}_{\text{bas}}|H
\end{equation}
As an alternative, sediment production could also be proportional to the product of basal velocities and basal shear stresses, $\vec{\tau}$, i.e.
\begin{equation}
  \dot{E} = f\vec{u}_{\text{bas}}\cdot\vec{\tau}.
\end{equation}
The total amount of eroded bedrock is then
\begin{equation}
  E=\int_{0}^t\dot{E}dt'
\end{equation}
\section{Isostatic Adjustment}
Sediment movement affects isostatic adjustment. This isostatic effect is taken into account by modifying the unloaded (no ice sheet) equilibrium bedrock topography. $h_0$:
\begin{equation}
  h_0=h_0^\ast+E-w_{\text{seds}}
\end{equation}
where $w_{\text{seds}}$ is the equilibrium bedrock depression due to erosion. $w_{\text{seds}}$ can be calculated using either the local or elastic lithosphere approximation (see my thesis).
