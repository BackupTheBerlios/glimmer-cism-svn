\subsection{Using \texttt{plotProfile.py}}
\texttt{plotProfile.py} can be used to plot variables along a profile line. Again, time slices can be selected and multiple variables plotted. A number of files can be plotted if the variable is not three--dimensional. If the variable is three--dimensional, like temperature, it is transformed from the $\sigma$-- to the $z$--coordinate system (see Chapter \ref{num.sec.sigma}).

\begin{pycf}{plotProfile.py -vis  -t -17 -p prof --not\_p fenscan.nc fenscan-gthf.nc is.ps}{\dir/figures/prof_is.eps}
  plots a profile of the ice elevation and bed rock topograpy at time -17ka.
\end{pycf}

\begin{pycf}{plotProfile.py -vis -vbvel -vbmelt -t -17 -p prof --not\_p $\backslash$ \newline   fenscan.nc fenscan-gthf.nc vars.ps}{\dir/figures/prof_vars.eps}
  stacks a number of variables on top of each other.
\end{pycf}

\begin{pycf}{plotProfile.py -vtemp  -t -18 -p prof --not\_p fenscan.nc temp.ps}{\dir/figures/prof_temp.eps}
  plots a profile of the internal temperature structure at -18ka.
\end{pycf}

\begin{pycf}{plot3DProfiles.py -vtemp  -t -18 -p prof --not\_p fenscan.nc fenscan-gthf.nc $\backslash$ \newline temp2.ps}{\dir/figures/prof_temp2.eps}
is a variation of \texttt{plotProfile.py} and can be used to plot vertical slices through a 3D field for different runs.
\end{pycf}
