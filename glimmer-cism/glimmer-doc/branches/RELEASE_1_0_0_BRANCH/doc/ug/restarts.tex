\section{Restarting the model}
%
This section describes the optional part of the Glimmer model code
that can be used to write a snapshot of the complete model state to
file on demand, and then restart the model from that state.

Users should note that while this code works to the best of my
knowledge, I would {\bf strongly advise} that you evaluate it with
some thoroughness before using it in `production' situations.
%
\subsection{Building the restart code}
%
Generation/compilation of the code is optional. To enable it, you need 
to run configure with \texttt{--enable-restarts}, as described in
section \ref{ug.sec.tarball} above.

When building the restart code, a Python program analyses the Fortran
code, and generates two files: \texttt{glimmer\_rst\_body.inc}  and 
\texttt{glimmer\_rst\_head.inc}. These contain all the code needed to
perform restarts on all model variables.
%
\subsection{Writing restart files}
%
Here's how to write a restart file:
\begin{enumerate}
\item Each restart file can (at present) only store one time-slice of data. This 
could be changed relatively easily, if that was desired.
\item To write a restart file, you first need to declare and open it, using these 
fortran commands (placed in the appropriate places):
\begin{verbatim}
  use glimmer_restart_common

  type(restart_file) :: restart

  restart=open_restart_file(<filename>,RESTART_CREATE)
\end{verbatim}
Here, \texttt{<filename>} is the name of the file you want to
create. The form
\texttt{<something>.restart.nc} is
recommended. \texttt{RESTART\_CREATE} is a module-defined parameter
that tells \texttt{open\_restart\_file} to create a new file, rather
than opening an existing file for reading.

\item Writing the data to file is accomplished with a single subroutine call. For 
GLINT, this is:
\begin{verbatim}
  call glint_write_restart(ice_sheet,restart)
\end{verbatim}
Here, \texttt{ice\_sheet} is the glint instance, and \texttt{restart} is the file opened above. 
For GLIDE, you need:
\begin{verbatim}
  call glide_write_restart(model,restart)
\end{verbatim}
where model is the GLIDE instance, and restart is the restart file instance.

\item Finally, you should close the restart file, like this:
\begin{verbatim}
  call close_restart_file(restart)
\end{verbatim}
\end{enumerate}
%
\subsection{Reading restart files}
%
Reading back the file is similarly straightforward. For GLINT, this is what 
you need:
\begin{verbatim}
  restart=open_restart_file(<filename>,RESTART_READ)
  call glint_read_restart(ice_sheet,restart)
  call close_restart_file(restart)
\end{verbatim}
Whereas, with GLIDE, you need:
\begin{verbatim}
  restart=open_restart_file(<filename>,RESTART_READ)
  call glide_read_restart(model,restart)
  call close_restart_file(restart)
\end{verbatim}
\texttt{<filename>}, \texttt{model}, \texttt{ice\_sheet} and restart are all as above. Note that the 
contents of the model instance variables \texttt{model} and \texttt{ice\_sheet} are 
overwritten by whatever is in the restart file.
%
\subsection{Dealing with restarted output files}
%
In implementing restarts, a decision had to be made about how to deal with 
output files from the restarted model. The restart file contains details of 
the output netcdf files that the model was configured to write, but the 
restarted model would need new output files opening. It was decided to use the 
information from the restart data to open a new set of output files, but with 
the prefix `\texttt{RESTART\_}' added to the filenames. Thus, the output files from the 
restarted model contain all the same fields as the old output files, 
beginning from the point where the restart file was written. If desired, the 
output prefix can be changed from `\texttt{RESTART\_}' by the specification of an 
optional third argument in \texttt{glint\_read\_restart} and \texttt{glide\_read\_restart}, e.g.:
\begin{verbatim}
  call glint_read_restart(ice_sheet,restart,'NEW_PREFIX_')
  call glide_read_restart(model,restart,'NEW_PREFIX_')
\end{verbatim}
