%
\section{GLINT}
This appendix details the subroutine calls provided by GLINT, and their
arguments. Note that where a type is given as \texttt{real(rk)}, this
indicates that the kind of the real type is specified by the value of
the parameter \texttt{rk}, which may be altered at compile-time (see appropriate
other documentation for details).
%
%%%%%%%%%%%%%%%%%%%%%%%%%%%%%%%%%%%%%%%%%%%%%%%
% INITIALISE_GLINT                            %
%%%%%%%%%%%%%%%%%%%%%%%%%%%%%%%%%%%%%%%%%%%%%%%
%
\subsection{Subroutine \texttt{initialise\_glint}}
%
\paragraph{Purpose} To initialise the ice model, and load in all relevant parameter files.
%
\paragraph{Name and mandatory arguments}
%
\begin{verbatim}
  subroutine initialise_glint(params,lats,longs,paramfile,time_step)
\end{verbatim}
%
\paragraph{Arguments}
%
\begin{center}
  \tablefirsthead{%
    \hline
  }
  \tablehead{%
    \hline
    \multicolumn{4}{|p{\textwidth}|}{\emph{\small continued from previous page}}\\
    \hline
  }
  \tabletail{%
    \hline
    \multicolumn{4}{|r|}{\emph{\small continued on next page}}\\
    \hline}
  \tablelasttail{\hline}
  \begin{supertabular*}{\textwidth}{@{\extracolsep{\fill}}lllp{5.5cm}}
    \multicolumn{4}{|l|}{{\bf Mandatory}}\\
    \hline
    \texttt{params}    & \texttt{type(glint\_params)} & \texttt{intent(inout)} &
    Ice model to be configured \\
    \texttt{lats(:)}   & \texttt{real(rk)} & \texttt{intent(in)} & latitudinal
    location of grid-points in global data (given in $^{\circ}\mathrm{N}$)\\ 
    \texttt{longs(:)}  & \texttt{real(rk)} & \texttt{intent(in)} & longitudinal
    location of grid-points in global data (given in $^{\circ}\mathrm{E}$)\\ 
    \texttt{paramfile} & \texttt{character(*)} & \texttt{intent(in)} & name of
    top-level parameter file \\
    \texttt{time\_step} & \texttt{integer} & \texttt{intent(in)} & Intended
    calling time-step (hours) \\
    \hline
    \multicolumn{4}{|l|}{{\bf Optional}}\\
    \hline
    \texttt{latb(:)} & \texttt{real(rk)} & \texttt{intent(in)} & Latiudinal
    locations of grid-box boundaries (degrees). This array has one more
    element than \texttt{lats}. \\ 
    \texttt{lonb(:)} & \texttt{real(rk)} & \texttt{intent(in)} & Longitudinal
    locations of grid-box boundaries (degrees). This array has one more
    element than \texttt{longs}. \\
    \texttt{orog(:,:)} & \texttt{real(rk)} & \texttt{intent(out)} & The
    initial orography (m). \\
    \texttt{albedo} & \texttt{real(rk)} & \texttt{intent(out)} & The initial
    ice albedo field \\
    \texttt{ice\_frac} & \texttt{real(rk)} & \texttt{intent(out)} & The initial
    ice fraction \\
    \texttt{orog\_lats} & \texttt{real(rk)} & \texttt{intent(in)} &
    Latitudinal location of gridpoints for global orography output\\
    \texttt{orog\_longs} & \texttt{real(rk)} & \texttt{intent(in)} &
    Longitudinal location of gridpoints for global orography output\\
    \texttt{orog\_latb} & \texttt{real(rk)} & \texttt{intent(in)} & Locations
    of the latitudinal boundaries of the grid-boxes (orography)\\
    \texttt{orog\_lonb} & \texttt{real(rk)} & \texttt{intent(in)} & Locations
    of the longitudinal boundaries of the grid-boxes (orography)\\
    \texttt{output\_flag} & \texttt{logical} & \texttt{intent(out)} & Set to
    show outputs have been updated (provided for consistency with main
    \texttt{glint} subroutine).\\
  \end{supertabular*}
\end{center}
%
\paragraph{Additional notes}
%
\begin{itemize}
\item The ice model determines the size of the global domain from the sizes of
  the arrays \texttt{lats} and \texttt{longs}.
\item The latitudes contained in \texttt{lats} must be in descending order, so
  that $\mathtt{lats(i)}>\mathtt{lats(i+1)}$ for $1\leq \mathtt{i} \leq
  \mathtt{size(lats)}$.
\item The optional arguments \texttt{orog\_lats}, \texttt{orog\_longs},
  \texttt{orog\_latb}, and \texttt{orog\_lonb} may be used to define the frid
  on which the orography is output from GLINT. This is useful if the global
  model has spectral dynamics, and thus a higher-resolution orography is
  needed for greater accuracy when transforming to spectral space. These
  arguments may not be present in arbitrary combinations - only
  \texttt{orog\_lats}+\texttt{orog\_longs},
  \texttt{orog\_lats}+\texttt{orog\_longs}+\texttt{orog\_latb},
  \texttt{orog\_lats}+\texttt{orog\_longs}+\texttt{orog\_lonb}, and
  \texttt{orog\_lats}+\texttt{orog\_longs}+\texttt{orog\_latb}+\texttt{orog\_lonb}
  are permitted. Other combinations will generate a fatal error.
\end{itemize}
%
%%%%%%%%%%%%%%%%%%%%%%%%%%%%%%%%%%%%%%%%%%%%%%%
% GLINT                                       %
%%%%%%%%%%%%%%%%%%%%%%%%%%%%%%%%%%%%%%%%%%%%%%%
%
\subsection{Subroutine \texttt{glint}}
%
\paragraph{Purpose}
%
To perform temporal averaging of input fields, and, if necessary, down-scale
those fields onto local projections and perform an ice model time-step. Output
files may be appended to, and if optional arguments used, fields made
available for feedback.
%
\paragraph{Name and mandatory arguments}
%
\begin{verbatim}
  subroutine glint(params,time,temp,precip,orog)
\end{verbatim}
%
\paragraph{Arguments}
%
\begin{center}
  \tablefirsthead{%
    \hline
  }
  \tablehead{%
    \hline
    \multicolumn{4}{|p{\textwidth}|}{\emph{\small continued from previous page}}\\
    \hline
  }
  \tabletail{%
    \hline
    \multicolumn{4}{|r|}{\emph{\small continued on next page}}\\
    \hline}
  \tablelasttail{\hline}
  \begin{supertabular*}{\textwidth}{@{\extracolsep{\fill}}lllp{5.5cm}}
    \multicolumn{4}{|l|}{{\bf Mandatory}}\\
    \hline
    \texttt{params} & \texttt{type(glint\_params)} & \texttt{intent(inout)} &
    parameters for this run \\
    \texttt{time} & \texttt{integer} & \texttt{intent(in)} & Current model time
    (hours) \\
    \texttt{temp(:,:)} & \texttt{real(rk)} & \texttt{intent(in)} & Surface
    temperature field ($^{\circ}\mathrm{C}$) \\
    \texttt{precip(:,:)} & \texttt{real(rk)} & \texttt{intent(in)} & Precipitation field (mm/s) \\
    \texttt{orog(:,:)} & \texttt{real(rk)} & \texttt{intent(in)} & Global orography (m) \\
    \hline
    \multicolumn{4}{|l|}{{\bf Optional}}\\
    \hline
    \texttt{zonwind(:,:)} & \texttt{real(rk)} & \texttt{intent(in)} & Zonal
    component of the wind field ($\mathrm{ms}^{-1}$) \\
    \texttt{merwind(:,:)} & \texttt{real(rk)} & \texttt{intent(in)} & Meridional 
    component of the wind field ($\mathrm{ms}^{-1}$) \\
    \texttt{output\_flag} & \texttt{logical} & \texttt{intent(out)} & Set to show
    new output fields have been calculated after an ice-model time-step. If this
    flag is not set, the output fields retain their values at input. \\ 
    \texttt{orog\_out(:,:)} & \texttt{real(rk)} & \texttt{intent(inout)} & Output
    orography (m)\\ 
    \texttt{albedo(:,:)} & \texttt{real(rk)} & \texttt{intent(inout)} & Surface
    albedo \\
    \texttt{ice\_frac(:,:)} & \texttt{real(rk)} & \texttt{intent(inout)} &
    Fractional ice coverage \\
    \texttt{water\_in(:,:)} & \texttt{real(rk)} & \texttt{intent(inout)} & The
    input fresh-water flux (mm, over ice time-step). Essentially precip, but
    provided for consistency.\\
    \texttt{water\_out(:,:)} & \texttt{real(rk)} & \texttt{intent(inout)} & The
    output fresh-water flux (mm, over ice time-step). This is simply the ablation calculated by
    the model, scaled up to the global grid. It is up to the global model to
    then  deal with it (route it to the oceans, land scheme, etc.) Note that
    the precipitation fed to the model but which doesn't get incorporated into
    the ice sheet because it falls over the sea is returned in this field. \\ 
    \texttt{total\_water\_in} & \texttt{real(rk)} & \texttt{intent(inout)} &
    Area-integrated water flux in (kg)\\ 
    \texttt{total\_water\_out} & \texttt{real(rk)} & \texttt{intent(inout)} &
    Area-integrated water flux out (kg)\\
    \texttt{ice\_volume} & \texttt{real(rk)} & \texttt{intent(inout)} & Total ice volume (m$^3$)\\
  \end{supertabular*}
\end{center}
\paragraph{Additional notes}
%
\begin{itemize}
\item The sizes of all two-dimensional fields passed as arguments must be the
  same as that implied by the sizes of the arrays used to pass latitude and
  longitude information when the model was initialised using
  \texttt{initialise\_glint}. There is
  currently no checking mechanism in place for this, so using fields of the wrong size
  will lead to unpredictable results.
\item Zonal and meridional components of the wind are only required if the
  small-scale precipitation parameterization is being used (with
  \texttt{whichprecip} set to 2).
\item The output field arguments only return data relevant to the parts of the globe
  covered by the ice model instances. The fraction of each global
  grid-box covered by ice model instances may be obtained using the
  \texttt{glint\_coverage\_map} subroutine below. 
\item The output orography field is given as a mean calculated over the part
  of the grid-box covered by ice  model instances. Thus, to calculate the
  grid-box mean, the output fields should be multiplied point-wise by the
  coverage fraction. 
\item Albedo is currently fixed at 0.4 for ice-covered ground, and set to zero
  elsewhere. The albedo is given for the part of the global grid box covered
  by ice, not as an average of the part covered by the ice model. No attempt
  is made to guess the albedo of the parts of the ice model domain \emph{not}
  covered by ice.
\end{itemize}
%
\paragraph{Example interpretation of output fields}
%
Consider a particular point, $(i,j)$ in the global domain. Suppose value
returned by \texttt{glint\_coverage\_map} for this point is 0.7, and the
output fields have these values:
\begin{verbatim}
  orog_out(i,j)  = 200.0
  albedo(i,j)    =   0.4
  ice_frac(i,j)  =   0.5
\end{verbatim}
%
What does this mean? Well, the ice model covers 70\% of the grid-box, and in
that part the mean surface elevation is 200\,m. Of the part covered by the ice
model, half is actually covered by ice. Thus, 35\% ($0.5\times 0.7$) of the global grid-box is
covered by ice, and the ice has an mean albedo of 40\%. The model makes no suggestion for the
albedo or elevation of the other 65\% of the grid-box. Currently, ice albedo
is a constant that may be changed in the appropriate configuration file, but
this output field is provided against the possibility that the model may be
extended at some point to include a model of ice albedo.
%
%%%%%%%%%%%%%%%%%%%%%%%%%%%%%%%%%%%%%%%%%%%%%%%
% END_GLINT                                   %
%%%%%%%%%%%%%%%%%%%%%%%%%%%%%%%%%%%%%%%%%%%%%%%
%
\subsection{Subroutine \texttt{end\_glint}}
%
\paragraph{Purpose} To perform general tidying-up operations, close files, etc.
%
\paragraph{Name and mandatory arguments}
%
\begin{verbatim}
  subroutine end_glint(params)
\end{verbatim}
%
\paragraph{Arguments}
%
\begin{center}
  \tablefirsthead{%
    \hline
  } 
  \tablehead{%
    \hline
    \multicolumn{4}{|p{\textwidth}|}{\emph{\small continued from previous page}}\\
    \hline
  } 
  \tabletail{%
    \hline
    \multicolumn{4}{|r|}{\emph{\small continued on next page}}\\
    \hline}
      \tablelasttail{\hline}
        \begin{supertabular*}{\textwidth}{@{\extracolsep{\fill}}lllp{5.5cm}}
          \texttt{params} & \texttt{type(glint\_params)} & \texttt{intent(inout)} & Ice model paramters \\
\end{supertabular*}
\end{center}
%
%%%%%%%%%%%%%%%%%%%%%%%%%%%%%%%%%%%%%%%%%%%%%%%
% GLINT_COVERAGE_MAP                          %
%%%%%%%%%%%%%%%%%%%%%%%%%%%%%%%%%%%%%%%%%%%%%%%
%
\subsection{Function \texttt{glint\_coverage\_map}}
%
\paragraph{Purpose} To obtain a map of fractional coverage of global
grid-boxes by the ice model instances. The function returns a value
indicating success, or giving error information.
%
\paragraph{Type, name and mandatory arguments}
%
\begin{verbatim}
  integer function glint_coverage_map(params,coverage,cov_orog)
\end{verbatim}
%
\paragraph{Arguments}
%
\begin{center}
  \tablefirsthead{%
    \hline
  } 
  \tablehead{%
    \hline
    \multicolumn{4}{|p{\textwidth}|}{\emph{\small continued from previous page}}\\
    \hline
  } 
  \tabletail{%
    \hline
    \multicolumn{4}{|r|}{\emph{\small continued on next page}}\\
    \hline}
      \tablelasttail{\hline}
        \begin{supertabular*}{\textwidth}{@{\extracolsep{\fill}}lllp{5.5cm}}
        \texttt{params} & \texttt{type(glint\_params)} & \texttt{intent(in)} & Ice model parameters \\
\texttt{coverage(:,:)} & \texttt{real(rk)} & \texttt{intent(out)} & Coverage
map (all fields except orography) \\
\texttt{cov\_orog(:,:)} & \texttt{real(rk)} & \texttt{intent(out)} & Coverage
map (orography) \\
\end{supertabular*}
\end{center}
%
\paragraph{Returned value}
%
\begin{center}
\begin{tabular}{cl}
\hline
Value & Meaning \\
\hline
\hline
0 & Coverage maps have been returned successfully \\
1 & Coverage maps not yet calculated; must call \texttt{initialise\_glint}
first \\
2 & Arrays \texttt{coverage} or \texttt{cov\_orog} are the wrong size \\
\hline
\end{tabular}
\end{center}
