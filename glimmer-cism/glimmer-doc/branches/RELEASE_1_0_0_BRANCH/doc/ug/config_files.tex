\section{Glimmer configuration files}
\label{ug.sec.cfile}
%
All Glimmer components are configured using a common file
format. While the format was developed specifically for Glimmer, it
has similarities to Windows \texttt{.ini} files. The code used to read
Glimmer configuration files is self-contained within the model, and
could easily be used elsewhere.

Configuration files are plain text files; by convention, they are
identified by the \texttt{.config} filename extension
(e.g. \texttt{greenland.config}). The file is divided into named
sections, which begin with the name of the section in square brackets
(e.g. \texttt{[grid]}). Each section contains a number of key-value
pairs. Keys are separated from their associated values by a \texttt{=}
or \texttt{:}. Note that lines, or lines starting with a \texttt{\#},
\texttt{;} or \texttt{!} are ignored.

An example section with some keys and values might be as follows:
%
\begin{verbatim}
[grid]
ewn = 76 
nsn = 141
upn = 11
\end{verbatim}

Sections and keys are case sensitive and may contain white
space. However, the configuration parser is very simple and thus the
number of spaces within a key or section name also matters. Sensible
defaults are used when a specific key is not found. 
