\section{Configuration File Parser}\label{dg.sec.config_file}
The run--time behaviour of the ice sheet model is controlled by configuration files. The old file format is based on Fortran namelists. The new configuration file format is loosely based on the format of Windows \texttt{.ini} files with sections containing name/value pairs. The new format is more flexible and can be easily understood by reading the configuration files. This section contains a description of the configuration file parser API.

\subsection{File Format}
The parser assumes a maximum number of 250 characters per line. Leading and trailing white space is ignored. Names are case sensitive.
\begin{description}
\item[Comments:] Empty lines and lines starting with \texttt{!}, \texttt{;} or \texttt{\#} are ignored.
\item[Sections:] Section names are enclosed with square prackets, \texttt{[]} and can be 20 character long.
\item[Parameters:] Parameter names are separated from their associated values with either \texttt{:} or \texttt{=}. The names can be 20 characters long. Values can be 200 characters long.
\end{description}

An example configuration file:
\begin{verbatim}
;a comment
[a section]
an_int  : 1
a_float = 2.0
a_char  = hey, this is rather cool
an_array = 10. 20. -10. 40. 100.

[another section]
! more comments
foo : bar
\end{verbatim}

\subsection{Architecture Overview}
The configuration data is stored as linked list. Each section is described by the following list element:
\begin{verbatim}
  type ConfigSection
     character(len=namelen) :: name
     type(ConfigValue), pointer :: values=>NULL()
     type(ConfigSection), pointer :: next=>NULL()
  end type ConfigSection
\end{verbatim}
The parameter name/value pairs defined in each section are stored in another linked list:
\begin{verbatim}
  type ConfigValue
     character(len=namelen) :: name
     character(len=valuelen) :: value
     type(ConfigValue), pointer :: next=>NULL()
  end type ConfigValue
\end{verbatim}
These linked lists are setup and read using subroutines.

\subsection{API}
\begin{description}
\item[Reading configuration files] Configuration files are read using \texttt{ConfigRead}. This subroutine parses the configuration file and populates the linked lists.
\begin{verbatim}
subroutine ConfigRead(fname,config)
  character(len=*), intent(in) :: fname
  type(ConfigSection), pointer :: config
end subroutine ConfigRead
\end{verbatim}
The pointer \texttt{config} contains the first section of the configuration file.
\item[Dumping configuration] The subroutine \texttt{PrintConfig} traverses the linked lists and prints them to standard output.
\begin{verbatim}
subroutine PrintConfig(config)
  type(ConfigSection), pointer :: config
end subroutine PrintConfig(config)
\end{verbatim}
\item[Searching for a Section] The subroutine \texttt{GetSection} can be used to find a specific section.
\begin{verbatim}
subroutine GetSection(config,found,name)
  type(ConfigSection), pointer :: config
  type(ConfigSection), pointer :: found
  character(len=*),intent(in) :: name
end subroutine GetSection
\end{verbatim}
On exit the pointer \texttt{found} will point to the first section called \texttt{name}. \texttt{found} points to \texttt{NULL()} if the section \texttt{name} is not found.
\item[Reading parameters] Paramter name/value pairs are found using the \texttt{GetValue} family of subroutines. \texttt{GetValue} provides an interface to the individual subroutines \texttt{GetValueChar}, \texttt{GetValueInt}, \texttt{GetValueReal}, \texttt{GetValueIntArray} and \texttt{GetValueRealArray}.
\begin{verbatim}
subroutine GetValue(section,name,val)
  type(ConfigSection), pointer :: section
  character(len=*),intent(in) :: name
  sometype :: val
  integer,intent(in), optional :: numval
end subroutine GetValue
\end{verbatim}
\texttt{section} is the section that should be searched for the parameter \texttt{name}. On exit \texttt{val} contains the parameter value if it is found, otherwise it is unchanged. 

The array versions of \texttt{GetValue} expect value to be a pointer to a one--dimensional array. \texttt{val} is deallocated if it was allocated on entry. The array versions of \texttt{GetValue} also accept an optional value, \texttt{numval}, with which the maximum number of array elements can be set. The default is 100. Array elements are separated by white space.
\end{description}
