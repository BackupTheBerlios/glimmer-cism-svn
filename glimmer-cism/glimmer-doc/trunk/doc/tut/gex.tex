
\section{EISMINT: using \texttt{glimmer-example}}
As you hopefully already know, the heart of GLIMMER is the actual ice sheet model
GLIDE. This is where ice physics are resolved etc. To model an ice sheet using
GLIDE, you at least need to provide it with information about the 
mass balance. To get you started with a real simple example climate driver,
download \texttt{glimmer-example} from the project homepage or via CVS, cd into
the directory and type
\begin{verbatim}
glide_launch.py example.config
\end{verbatim}
this will kick off a simple EISMINT-1 moving margin type model run. The results
are written to \texttt{example.nc}, use a viewer like ncview to visualise them.
Take a look at the \texttt{example.config} file printed below and read the documentation on
the EISMINT type climate driver (section \ref{driver:eismint}) to better understand what is happening:\\

\begin{verbatim}
# configuration for the EISMINT-1 test-case # moving margin

[EISMINT-1 moving margin]

[grid]
# grid sizes
ewn = 31
nsn = 31
upn = 11
dew = 50000
dns = 50000

[options]
temperature = 1
flow_law = 2
isostasy = 0
sliding_law = 4
marine_margin = 2
stress_calc = 2
evolution = 2
basal_water = 2
vertical_integration = 0

[time]
tend = 200000.
dt = 10.
ntem = 1.
nvel = 1.
niso = 1.

[parameters]
flow_factor = 1
geothermal = -42e-3

[CF default]
title: EISMINT-1 moving margin

[CF output]
name: example.nc
frequency: 1000
variables: thk uflx vflx bmlt temp
uvel vvel wvel
\end{verbatim}

The line \texttt{[EISMINT-1 moving margin]} sets the model type for this run to be EISMINT (simple_glide binary).
This can also be achieved by specifying the correct binary using the \texttt{-m} flag, e.g.
\begin{verbatim}
glide_launch.py -m simple_glide example.config
\end{verbatim}
It is probably advisable to use the \texttt{-m} option instead of specifying the binary using a 
keyterm, as this will only work for EIS and EISMINT model types. For ease of use, the option was integrated
in the config file for this example. 

The \texttt{[grid]} section sets up the topography for the model run.\\
As this is an EISMINT testcase, there is no 'real' input topography, but ice is building up on a
flat surface, which is why nothing more but the grid dimensions need to be specified. Be aware that
this only works for EISMINT type model runs using simple\_glide.
In this case, the mass balance is parameterised as a function of distance
from the grid center, resulting in a point symmetric ice sheet.
The grid used here has a size of 31x31 cells (\texttt{ewn x
nsn}), comprises of 11 vertical layers (\texttt{upn = 11}) and an internal cell spacing of
5000 (\texttt{dew} and \texttt{dns}).

The \texttt{[options]} sections determines the basic behaviour of the model:\\
\texttt{temperature = 1} resolves the temperature over the whole of the 11
layers of ice (instead of assuming ice to be isothermal), \texttt{isostasy = 0}
turns off the isostasy component, etc. (check the documentation).

In the \texttt{[time]} section, the end time of the model run is set to 200000
with a timestep size of 10 and keeping all internal update processes
(temperature and velocity) in line with the timesteps by setting their
multiplier to 1.

Flow factor and geothermal heat flux parameters are set in the
\texttt{[parameters]} section.

Finally, in the \texttt{[CF output]} section, the name of the file to store the results
is given, together with the \texttt{variables} that should be dumped to the file and the
\texttt{frequency} with which they are written to it (every 1000 years). In this example, 
ice thickness (\texttt{thk}), basal melt temperature \texttt{bmlt}, ice temperature \texttt{temp} etc 
is output to the result file every 1000 years. Note that this output frequency is independent of the
modelling timesteps.

You might want to try and change some of the parameters, e.g. speed up ice flow by 
increasing the flow factor, and re-run the model to see what happens.
This is fairly simple and straight forward example of how to get GLIMMER to do some 
basic modelling. If you want to see a bit more of what GLIMMER can do, try the next section.
