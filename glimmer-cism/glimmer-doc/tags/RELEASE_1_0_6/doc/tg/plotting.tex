\section{Visualising GLIMMER Output}
All PyCF tools share some of the options. You can select a variable to be processed with the \texttt{-v} switch. A time slice is selected using the \texttt{-t} or \texttt{-T} switch. The difference between the two switches is that you select a time (in ka) with the \texttt{-t} switch and a time slice (a number starting at 0) with the \texttt{-T} switch. You can also use negative numbers for the \texttt{-T} switch to start counting at the end, e.g. \texttt{-T-1} selects the last time slice.

Variables to be plotted are selected with the \texttt{-v} switch. The name of the variable is the same as in the netCDF file (see Appendix \ref{ug.sec.varlist}). There are some additional variables which are calculated from the variables stored in the netCDF file (see Table \ref{tg.tab.extra_vars}). If you select a three--dimensional variable, such as the ice temperature \texttt{temp}, then you can also specify which $\sigma$--level you want to plot using the \texttt{-l} option.

\begin{table}[htbp]
  \centering
  \begin{tabular}{|p{0.1\textwidth}|p{0.8\textwidth}|}
    \hline
    \texttt{is} & ice surface, $h+H$ \\
    \texttt{vel} & magnitude of horizontal velocity field, $\sqrt{u_x^2+u_y^2}$ \\
    \texttt{bvel} & magnitude of horizontal basal velocities \\
    \texttt{bvel\_tavg} & magnitude of horizontal basal velocities overaged over time.\\
    \texttt{\_avg} & append \texttt{\_avg} to 3D variables to calculate the vertical average of the variable.\\
    \hline
  \end{tabular}
  \caption{Additional variables calculated from variables in netCDF file.}
  \label{tg.tab.extra_vars}
\end{table}

For some plots you can select more than one variable or time or use a number of input files. You can change the width of a plot with the \texttt{--width} switch. This is particularly useful if you are plotting more than one variable/time/file with \texttt{plotCFvar.py}.

All PyCF tools produce postscript files. You can send these files straight to a postscript printer. If you want to include images in a {\LaTeX} document you need to convert them to an encapsulated postscript file. \texttt{ps2epsi} is such a utility and is used for the examples shown here. You can also use \texttt{convert} which is part of \href{http://www.imagemagick.org/script/index.php}{ImageMagick} to convert the resulting postscript files to any other graphics format such as \texttt{PNG}.

You can always use the switch \texttt{-h} or \texttt{--help} to get some documentation of the switches. Some of the options have no effect or do not behave as you would expect. Please expirment with the tools and report bugs and make suggestions for improvement. 

\section{Using \texttt{plotCFvar.py}}
\texttt{plotCFvar.py} is used to plot a time slice of a 2D variable, such as ice thickness or basal velocities. Furthermore, a vertical slice can be specified if the field is three--dimensional such as the temperature or full velocity field. The displayed data can be clipped to the ice extent or area above sea--level. A three--dimensional effect can be created by illuminating the result.

\begin{pycf}{plotCFvar.py -vtopg fenscan.nc topo.ps}{\dir/figures/topo.eps}
plots the bedrock topgraphy. The \texttt{-v} switch selects the variable. The first time slice is processed if no time slice is specified on the command line. 

The GMT coastline database is used to add present--day coastlines. The colourmap is selected for a set of predefined colourmaps.
\end{pycf}

\begin{pycf}{plotCFvar.py -vis -cthk -iis -t -17 --land fenscan.nc is.ps}{\dir/figures/is.eps}
plots the ice surface which is calculated from the variables \texttt{topg} and \texttt{thk}. The switch \texttt{-t} select a time slice at -17ka. The data is clipped to the area covered by ice with the \texttt{-c} switch. \texttt{-i} adds illumination. The \texttt{--land} switch colours areas above sea--level in grey.
\end{pycf}

\begin{pycf}{plotCFvar.py -vbtemp -cthk -gbvel --pmt -t -17 --land fenscan.nc btemp.ps}{\dir/figures/btemp.eps}
plots the basal temperatures adjusted for its dependance on pressure (\texttt{--pmt}) and adds glyphs for the basal velocity directions with the \texttt{-g} switch. The density of glyphs is automatically adjusted to the size of the plot. Glyphs only indicate direction. If you want magnitudes as well you should just plot the basal velocity field with \texttt{-vbvel}.
\end{pycf}

\begin{pycf}{plotCFvar.py -vbvel --urg=20 60 -t -17 fenscan.nc bvel.ps}{\dir/figures/bvel.eps}
plots an enlargement of the basal velocity field. You can specify the geographic coordinates of the lower--left corner (\texttt{--llg}) and/or upper--right corner (\texttt{--urg}) to define the region to be plotted. Alternatively, you can also specify the corners in the projected coordinate system using \texttt{--llx} and \texttt{--urx}.
\end{pycf}

\begin{pycf}{plotCFvar.py -vbvel -cthk -t -17 --land -pprof --not\_p}{\dir/figures/bvelp.eps}
plots a 2D field with a profile line. The \texttt{-p} selects a file containing coordinates of the profile. This option can be repeated to have more than one line. The \texttt{--not\_p} is short for \texttt{--not\_projected} and indicates that the coordinates have to be projected to the map coordinate system. Profile lines are labeled automatically, starting with A.
\end{pycf}

\subsection{Using \texttt{plotStreams.py}}
\texttt{plotStreams.py} is similar to \texttt{plotCFvar.py} with the only difference that it plots areas where the ice base is sliding over a given time interval. The time interval can be set with the \texttt{--deltat} option. Basal velocities are processed for the time interval centred at the time slice selected with \texttt{-t} or \texttt{-T}.

\begin{pycf}{plotStreams.py -t-10 --deltat=10000 --legend fenscan.nc streams1.ps}{\dir/figures/streams1.eps}
plots relative time an area is occupied by sliding ice during the interval -20ka to 0ka, i.e. a value of 1 indicates that the cell was sliding and a value of 0 indicates that ice was frozen to the bed  through out the interval. The \texttt{--legend} switch adds the colour legend.
\end{pycf}

\begin{pycf}{plotStreams.py -t-10 --deltat=10000 --velocity fenscan.nc streams2.ps}{\dir/figures/streams2.eps}
plots mean basal velocities during the interval -20ka to 0ka.
\end{pycf}

\section{Using \texttt{plot\_extent.py}}
\texttt{plot\_extent.py} plots the ice extent at a given time. This program is particularly useful to compare different runs. \texttt{plot\_extent.py} can also plot profile lines using \texttt{-p} switch like \texttt{plotCFvar.py}.

\begin{pycf}{plot\_extent.py -t -17 fenscan.nc fenscan-gthf.nc extent.ps}{\dir/figures/extent.eps}
plots the ice extent of simulations stored in \texttt{fenscan.nc} and \texttt{fenscan-gthf.nc} at time -17ka.
\end{pycf}

\section{Using \texttt{plotProfile.py}}
\texttt{plotProfile.py} can be used to plot variables along a profile line. Again, time slices can be selected and multiple variables plotted. A number of files can be plotted if the variable is not three--dimensional. If the variable is three--dimensional, like temperature, it is transformed from the $\sigma$-- to the $z$--coordinate system (see Chapter \ref{num.sec.sigma}).

\begin{pycf}{plotProfile.py -vis  -t -17 -p prof --not\_p fenscan.nc fenscan-gthf.nc is.ps}{\dir/figures/prof_is.eps}
  plots a profile of the ice elevation and bed rock topograpy at time -17ka.
\end{pycf}

\begin{pycf}{plotProfile.py -vis -vbvel -vbmelt -t -17 -p prof --not\_p $\backslash$ \newline   fenscan.nc fenscan-gthf.nc vars.ps}{\dir/figures/prof_vars.eps}
  stacks a number of variables on top of each other.
\end{pycf}

\begin{pycf}{plotProfile.py -vtemp  -t -18 -p prof --not\_p fenscan.nc temp.ps}{\dir/figures/prof_temp.eps}
  plots a profile of the internal temperature structure at -18ka.
\end{pycf}

\subsection{Using \texttt{plotProfileTS.py}}
\begin{pycf}{plotProfileTS.py -t -30 0 -vbtemp --pmt -cthk -p prof --not\_p fenscan.nc td1.ps}{\dir/figures/time_dist1.eps}
plots a time--distance diagram of the variable selected with \texttt{-v}. You can specify the time interval with \texttt{-t}. All time slices are processed if you do not specify a time interval. As before the variable is clipped to the area covered by ice.
\end{pycf}

\begin{pycf}{plotProfileTS.py -e stages --profvar is -vbtemp --pmt -cthk -p prof --not\_p $\backslash$ \newline fenscan.nc td2.ps}{\dir/figures/time_dist2.eps}
You can specify a variable to be plotted at the beginning and end of the time interval using the \texttt{--profvar} option. With the \texttt{-e} option you can plot the time scale on the right. The \texttt{stages} file contains 4 comma--separated columns. The first column contains the name, second and thrid column start and end time in years, and the last column a R/G/B triplet for the background colour. 
\end{pycf}

\subsection{Using \texttt{plotSpot.py}}
\texttt{plotSpot.py} is used to reduce the variable to one dimension by either plotting a time series of a 2D variable or a vertical profile of a 3D variable at a given time for a given location. The location is specified with a pair of integers $(i,j)$ identifying the node. The lower--left corner is $(0,0)$. 

\begin{pycf}{plotSpot.py --ij 57 42 --ij 77 76 -vtemp  -t-18 fenscan.nc temp.ps}{\dir/figures/spot_temp.eps}
plots ice temperatures as a function of $\sigma$ for two locations at time -18ka.  
\end{pycf}

\begin{pycf}{plotSpot.py --ij 77 76 -vthk  fenscan.nc fenscan-gthf.nc thk.ps}{\dir/figures/spot_thk.eps}
plots ice thickness as a function of time at one location for two different runs.
\end{pycf}

\subsection{Using \texttt{plotCFstats.py}}
\begin{pycf}{plotCFstats.py -mi -p profname --not\_p -f data fenscan.nc fenscan-gthf.nc stats.ps}{\dir/figures/stats.eps}
plots ice sheet statistics as a function of time. Single letter switches can be joined, e.g. the options \texttt{-m} and \texttt{-i} become \texttt{-mi} and switch on melt fractional area and average ice thickness. The ice extent along a profile is also calculated (\texttt{-p} option) and can be compared with a data file (\texttt{-f} switch).
\end{pycf}

\subsection{Using \texttt{plotCFdiff.py}}
\begin{pycf}{plotCFdiff.py fenscan.nc fenscan-gthf.nc diff.ps}{\dir/figures/diff.eps}
plots a graphical diff of two GLIMMER model runs. You can change the time interval using the \texttt{-t} or \texttt{-T} option. The three top panels show the difference of the ice volume, ice area and melt fractional area. The lower panels show 2D histograms of the difference to give you an idea of the distribution of differences.
\end{pycf}

\input{\dir/rsl.tex}

\subsection{Producing Animations}
PyCF includes a program which can be used to produce an animation. Any PyCF program that uses the \texttt{-T} option to select a time slice can be animated with \texttt{make\_anim.py}. \texttt{make\_anim.py} uses \href{http://www.mplayerhq.hu/homepage/design7/news.html}{\texttt{mencoder}} to combine the individual frames to an animation. You will also need ImageMagick to convert the postscript files to PNG files.

For example, the command
\begin{verbatim}
make_anim.py -t 0 120 "plotCFvar.py -vis -cthk -iis --land fenscan.nc"
\end{verbatim}
will produce an animation called \texttt{anim.avi} (you can override the output file name with the \texttt{-o} option) of the ice surface evolution. The \texttt{-t} option selects the range of time slices to be processed. The frame rate of the animation can be changed with the \texttt{-f} option. A different CODEC can be selected using \texttt{-c}. \texttt{mencoder} might fail to produce an animation. This is usually due to the fact that the CODEC can only handle certain size frames. It is therefore advisable to first tryout a short animation with 5 frames or so and to change the size with \texttt{-s} option if it fails. You can pass further options to \texttt{convert} using the \texttt{--convert\_options} option.
