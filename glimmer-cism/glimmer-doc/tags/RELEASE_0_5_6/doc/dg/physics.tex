
\section{Physics documentation}

\subsection{Ice temperature evolution routines}

\subsubsection{Summary}
Call structure (filenames in brackets).
\begin{itemize}
    \item subroutine testinisthk [glimmer\_setup] and
    \item subroutine glimmer\_i\_tstep [glimmer\_object] call
    \item subroutine timeevoltemp [glimmer\_temp] calls
    \item subroutine calcartm [glimmer\_temp] and
    \item subroutine timeders [glimmer\_thck] and
    \item subroutine gridwvel [glimmer\_velo] and
    \item subroutine wvelintg [glimmer\_velo] and
    \item subroutine chckwvel [glimmer\_velo] and
    \item subroutine finddisp [glimmer\_temp] and
    \item subroutine hadvall [glimmer\_temp] and
    \item subroutine hadvpnt [glimmer\_temp] and
    \item subroutine findvtri [glimmer\_temp] and
    \item subroutine tridag [glimmer\_temp] and
    \item subroutine corrpmpt [glimmer\_temp] and
    \item subroutine swapbndt [glimmer\_temp] and
    \item subroutine calcbmlt [glimmer\_temp] and
    \item subroutine calcflwa [glimmer\_temp]
\end{itemize}

\noindent Modules used.
\begin{itemize}
    \item
\end{itemize}

\subsubsection{Introduction}
The section describes the routines that are concerned with
calculating the three-dimensional distribution of temperature
within the ice mass.  They can be broken down into five groups.
\begin{itemize}
    \item determining air temperature (upper boundary
    condition) [\texttt{calcartm}];
    \item determining vertical velocity field from existing
    horizontal velocity fields (normally only needed if temperature is being calculated) [\texttt{wvelintg}, chckwvel];
    \item routines associated with vertical grid coordinate
    system [\texttt{gridwvel}, \texttt{timeders}];
    \item the main temperature solver [\texttt{finddisp, hadvall, hadvpnt, findvtri, tridag, corrpmpt, swapbndt}];
    \item ancillary calculations that only make sense if temperature is being calculated
    [\texttt{calcbmlt}, \texttt{calcflwa}].
\end{itemize}

The basic quantity returned is a three-dimensional grid of
temperature in $\circ^{-1}$C (uncorrected for variations in
pressure melting point and unscaled).  Temperature is held in the
array \texttt{temp} and will be referred to here using the symbol
$T$.

In addition to temperature a number of other quantities are
calculated by these routines.  They include: basal melt rate ($m$
\texttt{bmlt} m yr$^{-1}$ scaled using \texttt{thk0/tim0}); basal
water depth ($W$ \texttt{bwat} m scaled using \texttt{thk0});
vertical velocity ($w$ \texttt{wvel} m yr$^{-1}$ scaled using
\texttt{thk0/tim0}); vertical velocity of numerical grid ($w_0$
\texttt{wgrd} m yr$^{-1}$ scaled using \texttt{thk0/tim0}); Glen's
A ($A$ \texttt{flwa} Pa$^{-3}$ yr$^{-1}$ scaled using
\texttt{vis0}); air temperature ($T_a$ $\circ^{-1}$C unscaled).
All scales are held in the module \texttt{paramets} in
\textbf{\texttt{glimmer\_paramets}}.

Three options are currently available for calculating $T$. The
particular option chosen is controlled by the input parameter
\texttt{whichtemp} (\texttt{gln} file).

\begin{description}
    \item[0] Set whole column to the appropriate surface air temperature ($T_a$).
    \item[1] This option is the main solver that determines temperature
    at the new time step from the appropriate three-dimensional
    advection-diffusion equation.
    \item[2] Set the upper surface temperature to $T_a$ and do a linear
    interpolation from this value to 0 $^\circ$C at the lower
    surface. Check for pressure melting and adjust any
    temperatures that are above melting point.
\end{description}

The subroutine \texttt{timeevoltemp} controls calculation of the
$T$ etc. It is called in the main time loop in
\textbf{\texttt{glimmer\_object}} and resides in
\textbf{\texttt{glimmer\_temp}}.