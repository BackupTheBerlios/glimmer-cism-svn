\section{Isostatic Adjustment}
The ice sheet model includes simple approximations for calculating isostatic adjustment. These approximations depend on how the lithosphere and the mantle are treated. For each subsystem there are two models. The lithosphere can be described as a
\begin{description}
\item[\textbf{local lithosphere:}] the flexural rigidity of the lithosphere is ignored, i.e. this is equivalent to ice floating directly on the asthenosphere;
\item[\textbf{elastic lithosphere:}] the flexural rigidity is taken into account;
\end{description}
while the mantle is treated as a
\begin{description}
\item [\textbf{fluid mantle:}] the mantle behaves like a non-viscous fluid, isostatic equilibrium is reached instantaneously;
\item [\textbf{relaxing mantle:}] the flow within the mantle is approximated by an exponentially decaying hydrostatic response function, i.e. the mantle is treated as a viscous half space.
\end{description}

\subsection{Calculation of ice-water load}
At each isostasy time-step, the load of ice and water is calculated, as an
equivalent mantle-depth ($L$). If the basal elevation is above sea-level, then the
load is simply due to the ice:
\begin{equation}
L=\frac{\rho_i}{\rho_m}H,
\label{load_land_ice}
\end{equation}
where $H$ is the ice thickness, with $\rho_i$ and $\rho_m$ being the densities
of the ice and mantle respectively. In the case where the bedrock is below
sea-level, the load is calculated is that due to a change in sea-level rise and/or
the presence of non-floating ice. When the ice is floating ($\rho_i
H<\rho_o(z_0-h)$), the load is only due to sea-level changes
\begin{equation}
L=\frac{\rho_o}{\rho_m}z_0,
\label{load_sea_float}
\end{equation}
whereas when the ice is grounded, it displaces the water, and adds an
additional load:
\begin{equation}
L=\frac{\rho_i H+\rho_o h}{\rho_m}.
\label{load_sea_grounded}
\end{equation}
here, $\rho_o$ is the density of sea water, $z_0$ is the change in sea-level
relative to a reference level and $h$ is the bedrock elevation relative to the
same reference level. The value of $h$ will be negative for submerged bedrock,
hence the plus sign in (\ref{load_sea_grounded}).

\subsection{Elastic lithosphere model}
This is model is selected by setting \texttt{lithosphere = 1} in the
configuration file. By simulatuing the deformation of the lithosphere, the
deformation seen by the aesthenosphere beneath is calculated. In the absence of this
model, the deformation is that due to Archimedes' Principle, as though the
load were floating on the aesthenosphere.

The elastic lithosphere model is based on work by \cite{Lambeck1980}, and its
implementation is fully described in \cite{Hagdorn2003}. The lithosphere
model only affects the geometry of the deformation --- the timescale for
isostatic adjustment is controlled by the aesthenosphere model. 

The load due to a single (rectangular) grid point is approximated as being
applied to a disc of the same area. The deformation due to a disc of ice of
radius $A$ and thickness $H$ is given by these expressions. For $r<A$:
\begin{equation} 
w(r)=\frac{\rho_i H}{\rho_m}\left[1+C_1\,\mathrm{Ber}\left(\frac{r}{L_r}\right)+C_2\,\mathrm{Bei}\left(\frac{r}{L_r}\right)\right],
\end{equation}
and for $r\geq A$:
\begin{equation}
w(r)=\frac{\rho_i
  H}{\rho_m}\left[D_1\,\mathrm{Ber}\left(\frac{r}{L_r}\right)+D_2\,\mathrm{Bei}\left(\frac{r}{L_r}\right)
+D_3\,\mathrm{Ker}\left(\frac{r}{L_r}\right)+D_4\,\mathrm{Kei}\left(\frac{r}{L_r}\right)\right],
\end{equation}
where $\mathrm{Ber}(x)$, $\mathrm{Bei}(x)$, $\mathrm{Ker}(x)$ and
$\mathrm{Kei}(x)$ are Kelvin functions of zero order, $L_r=(D/\rho_m
g))^{1/4}$ is the radius of relative stiffness, and $D$ is the flexural
rigidity. The constants $C_i$ and $D_i$ are given by
\begin{equation}
\begin{array}{rcl}
C_1&=&a\,\mathrm{Ker}'(a)\\
C_2&=&-a\,\mathrm{Ker}'(a)\\
D_1&=&0\\
D_2&=&0\\
D_3&=&a\,\mathrm{Ber}'(a)\\
D_4&=&-a\,\mathrm{Ber}'(a).
\end{array}
\end{equation}
Here, the prime indicates the first spatial derivative of the Kelvin functions.

\subsection{Relaxing aesthenosphere model}
If a fluid mantle is selected, it adjusts instantly to changes in lithospheric
loading. However, a relaxing mantle is also available.

%%% Local Variables: 
%%% mode: latex
%%% TeX-master: "isos"
%%% End: 
